\pagenumbering{roman}
\hspace{0pt}
\vfill
\begin{center}

\Huge{Learning Statistics with jamovi}

\Large{A Tutorial for Beginners in Statistical Analysis}\\*[20pt]

\normalsize{Danielle Navarro and David Foxcroft}

\vfill
\end{center}
\hspace{0pt}
\pagebreak

\hspace{0pt}
\vfill

\copyright 2025 David Foxcroft and Danielle Navarro

This work is licensed under an Attribution-ShareAlike 4.0 International (CC BY-SA 4.0).

This license allows you to copy and redistribute, transform, and build upon the material for any purpose, even commercially. providing attribution is made to the authors (but not in any way that suggests that they endorse you or your use of the work). Attribution should include the following information:

Danielle Navarro and David Foxcroft, \textit{Learning statistics with jamovi: A tutorial for beginners in statistical analysis}. Cambridge, UK: Open Book Publishers, 2025, \url{https://doi.org/10.11647/OBP.0333}

Further details about CC BY-SA licenses are available at \\ \url{https://creativecommons.org/licenses/by-sa/4.0/}

All external links were active at the time of publication unless otherwise stated and have been archived via the Internet Archive Wayback Machine at \\ \url{https://archive.org/web}

Digital material and resources associated with this volume are available at \\ \url{https://doi.org/10.11647/OBP.0333\#resources}

ISBN Paperback: 978-1-80064-937-8

ISBN Hardback: 978-1-80064-938-5

ISBN Digital (PDF): 978-1-80064-939-2

DOI: 10.11647/OBP.0333


\vfill
\hspace{0pt}
\pagebreak

\hspace{0pt}
\vfill

This textbook covers the contents of an introductory statistics class, as typically taught to undergraduate psychology, health or social science students. The book covers how to get started in jamovi as well as giving an introduction to data manipulation. From a statistical perspective, the book discusses descriptive statistics and graphing first, followed by chapters on probability theory, sampling and estimation, and null hypothesis testing. After introducing the theory, the book covers the analysis of contingency tables, correlation, \textit{t}-tests, regression, ANOVA and factor analysis. Bayesian statistics are touched on at the end of the book.

Data sets used in the book are freely available for use in jamovi. All the data files you need can be accessed within jamovi via an add-on module in the jamovi library. Or you can download the files from \url{https://www.learnstatswithjamovi.com}.


Citation: Danielle Navarro and David Foxcroft, \textit{Learning statistics with jamovi: A tutorial for beginners in statistical analysis}. Cambridge, UK: Open Book Publishers, 2025, \url{https://doi.org/10.11647/OBP.0333}

\vfill
\hspace{0pt}
