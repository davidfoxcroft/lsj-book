% Options for packages loaded elsewhere
\PassOptionsToPackage{unicode}{hyperref}
\PassOptionsToPackage{hyphens}{url}
\PassOptionsToPackage{dvipsnames,svgnames,x11names}{xcolor}
%
\documentclass[
  a4paper,
]{book}

\usepackage{amsmath,amssymb}
\usepackage{iftex}
\ifPDFTeX
  \usepackage[T1]{fontenc}
  \usepackage[utf8]{inputenc}
  \usepackage{textcomp} % provide euro and other symbols
\else % if luatex or xetex
  \usepackage{unicode-math}
  \defaultfontfeatures{Scale=MatchLowercase}
  \defaultfontfeatures[\rmfamily]{Ligatures=TeX,Scale=1}
\fi
\usepackage{lmodern}
\ifPDFTeX\else  
    % xetex/luatex font selection
  \setmainfont[]{TeX Gyre Pagella}
\fi
% Use upquote if available, for straight quotes in verbatim environments
\IfFileExists{upquote.sty}{\usepackage{upquote}}{}
\IfFileExists{microtype.sty}{% use microtype if available
  \usepackage[]{microtype}
  \UseMicrotypeSet[protrusion]{basicmath} % disable protrusion for tt fonts
}{}
\makeatletter
\@ifundefined{KOMAClassName}{% if non-KOMA class
  \IfFileExists{parskip.sty}{%
    \usepackage{parskip}
  }{% else
    \setlength{\parindent}{0pt}
    \setlength{\parskip}{6pt plus 2pt minus 1pt}}
}{% if KOMA class
  \KOMAoptions{parskip=half}}
\makeatother
\usepackage{xcolor}
\usepackage[paperwidth=8.0000000000000in,paperheight=10.000000000000in,left=1.25in,textwidth=
5.25in,top=1.00in,textheight=8.25in]{geometry}
\setlength{\emergencystretch}{3em} % prevent overfull lines
\setcounter{secnumdepth}{5}
% Make \paragraph and \subparagraph free-standing
\ifx\paragraph\undefined\else
  \let\oldparagraph\paragraph
  \renewcommand{\paragraph}[1]{\oldparagraph{#1}\mbox{}}
\fi
\ifx\subparagraph\undefined\else
  \let\oldsubparagraph\subparagraph
  \renewcommand{\subparagraph}[1]{\oldsubparagraph{#1}\mbox{}}
\fi


\providecommand{\tightlist}{%
  \setlength{\itemsep}{0pt}\setlength{\parskip}{0pt}}\usepackage{longtable,booktabs,array}
\usepackage{calc} % for calculating minipage widths
% Correct order of tables after \paragraph or \subparagraph
\usepackage{etoolbox}
\makeatletter
\patchcmd\longtable{\par}{\if@noskipsec\mbox{}\fi\par}{}{}
\makeatother
% Allow footnotes in longtable head/foot
\IfFileExists{footnotehyper.sty}{\usepackage{footnotehyper}}{\usepackage{footnote}}
\makesavenoteenv{longtable}
\usepackage{graphicx}
\makeatletter
\def\maxwidth{\ifdim\Gin@nat@width>\linewidth\linewidth\else\Gin@nat@width\fi}
\def\maxheight{\ifdim\Gin@nat@height>\textheight\textheight\else\Gin@nat@height\fi}
\makeatother
% Scale images if necessary, so that they will not overflow the page
% margins by default, and it is still possible to overwrite the defaults
% using explicit options in \includegraphics[width, height, ...]{}
\setkeys{Gin}{width=\maxwidth,height=\maxheight,keepaspectratio}
% Set default figure placement to htbp
\makeatletter
\def\fps@figure{htbp}
\makeatother
\newlength{\cslhangindent}
\setlength{\cslhangindent}{1.5em}
\newlength{\csllabelwidth}
\setlength{\csllabelwidth}{3em}
\newlength{\cslentryspacingunit} % times entry-spacing
\setlength{\cslentryspacingunit}{\parskip}
\newenvironment{CSLReferences}[2] % #1 hanging-ident, #2 entry spacing
 {% don't indent paragraphs
  \setlength{\parindent}{0pt}
  % turn on hanging indent if param 1 is 1
  \ifodd #1
  \let\oldpar\par
  \def\par{\hangindent=\cslhangindent\oldpar}
  \fi
  % set entry spacing
  \setlength{\parskip}{#2\cslentryspacingunit}
 }%
 {}
\usepackage{calc}
\newcommand{\CSLBlock}[1]{#1\hfill\break}
\newcommand{\CSLLeftMargin}[1]{\parbox[t]{\csllabelwidth}{#1}}
\newcommand{\CSLRightInline}[1]{\parbox[t]{\linewidth - \csllabelwidth}{#1}\break}
\newcommand{\CSLIndent}[1]{\hspace{\cslhangindent}#1}

\usepackage{makeidx}
\usepackage{pdfpages}

%reduce vertical spacing in toc - not needed once tocloft added below
%\usepackage{etoolbox}
%\makeatletter
%  \pretocmd{\chapter}{\addtocontents{toc}{\protect\addvspace{-1\p@}}}{}{}
%  \pretocmd{\section}{\addtocontents{toc}{\protect\addvspace{-5\p@}}}{}{}
%\pretocmd{\subsection}{\addtocontents{toc}{\protect\addvspace{-5\p@}}}{}{}
%\makeatother

%\makeatletter
%  \providecommand*\setfloatlocations[2]{\@namedef{fps@#1}{#2}}
%\makeatother
%\setfloatlocations{figure}{htbp}
%\setfloatlocations{table}{htbp}

\usepackage{titling}
\let\oldmaketitle\maketitle

\usepackage{atbegshi}% http://ctan.org/pkg/atbegshi
\AtBeginDocument{\let\maketitle\relax}
\AtBeginDocument{\AtBeginShipoutNext{\AtBeginShipoutDiscard}} % Discard next blank page

\usepackage{tocloft}
\cftsetindents{section}{2em}{2.5em}
\cftsetindents{subsection}{5em}{3em}

\usepackage{enotez}
\DeclareInstance{enotez-list}{section}{paragraph}{heading=\chapter}
\setenotez{split=chapter,backref=true,list-name=Chapter notes,totoc=chapter}

%define command for adding footnote textbox to first page of each chapter
\newcommand{\placetextbox}[3]{% \placetextbox{<horizontal pos>}{<vertical pos>}{<stuff>}
  \setbox0=\hbox{#3}% Put <stuff> in a box
  \AddToShipoutPictureFG*{% Add <stuff> to current page foreground
    \put(\LenToUnit{#1\paperwidth},\LenToUnit{#2\paperheight}){\vtop{{\null}\makebox[0pt][c]{#3}}}%
  }%
}%

\makeatletter
\@beginparpenalty=10000
\makeatother

\usepackage[labelfont=bf]{caption}

\raggedbottom

\usepackage{array}
\usepackage{caption}
\usepackage{graphicx}
\usepackage{siunitx}
\usepackage[normalem]{ulem}
\usepackage{colortbl}
\usepackage{multirow}
\usepackage{hhline}
\usepackage{calc}
\usepackage{tabularx}
\usepackage{threeparttable}
\usepackage{wrapfig}
\usepackage{adjustbox}
\usepackage{hyperref}
\makeatletter
\makeatother
\makeatletter
\@ifpackageloaded{bookmark}{}{\usepackage{bookmark}}
\makeatother
\makeatletter
\@ifpackageloaded{caption}{}{\usepackage{caption}}
\AtBeginDocument{%
\ifdefined\contentsname
  \renewcommand*\contentsname{Table of contents}
\else
  \newcommand\contentsname{Table of contents}
\fi
\ifdefined\listfigurename
  \renewcommand*\listfigurename{List of Figures}
\else
  \newcommand\listfigurename{List of Figures}
\fi
\ifdefined\listtablename
  \renewcommand*\listtablename{List of Tables}
\else
  \newcommand\listtablename{List of Tables}
\fi
\ifdefined\figurename
  \renewcommand*\figurename{Figure}
\else
  \newcommand\figurename{Figure}
\fi
\ifdefined\tablename
  \renewcommand*\tablename{Table}
\else
  \newcommand\tablename{Table}
\fi
}
\@ifpackageloaded{float}{}{\usepackage{float}}
\floatstyle{ruled}
\@ifundefined{c@chapter}{\newfloat{codelisting}{h}{lop}}{\newfloat{codelisting}{h}{lop}[chapter]}
\floatname{codelisting}{Listing}
\newcommand*\listoflistings{\listof{codelisting}{List of Listings}}
\makeatother
\makeatletter
\@ifpackageloaded{caption}{}{\usepackage{caption}}
\@ifpackageloaded{subcaption}{}{\usepackage{subcaption}}
\makeatother
\makeatletter
\@ifpackageloaded{tcolorbox}{}{\usepackage[skins,breakable]{tcolorbox}}
\makeatother
\makeatletter
\@ifundefined{shadecolor}{\definecolor{shadecolor}{rgb}{.97, .97, .97}}
\makeatother
\makeatletter
\makeatother
\makeatletter
\makeatother
\ifLuaTeX
  \usepackage{selnolig}  % disable illegal ligatures
\fi
\IfFileExists{bookmark.sty}{\usepackage{bookmark}}{\usepackage{hyperref}}
\IfFileExists{xurl.sty}{\usepackage{xurl}}{} % add URL line breaks if available
\urlstyle{same} % disable monospaced font for URLs
\hypersetup{
  pdftitle={LEARNING STATISTICS WITH JAMOVI},
  colorlinks=true,
  linkcolor={Maroon},
  filecolor={Maroon},
  citecolor={Blue},
  urlcolor={Blue},
  pdfcreator={LaTeX via pandoc}}

\title{LEARNING STATISTICS WITH JAMOVI}
\author{}
\date{}

\begin{document}
\frontmatter
\maketitle
\begin{center}
\includepdf[fitpaper=true,pages=-]{images/obp.0333}
\end{center}
\pagestyle{empty}

\let\maketitle\oldmaketitle
\maketitle

\mainmatter
\pagestyle{plain}

\let\footnote=\endnote



\pagenumbering{roman}
\hspace{0pt}
\vfill
\begin{center}

\Huge{Learning Statistics with jamovi}

\Large{A Tutorial for Beginners in Statistical Analysis}\\*[20pt]

\normalsize{Danielle Navarro and David Foxcroft}

\vfill
\end{center}
\hspace{0pt}
\pagebreak

\hspace{0pt}
\vfill

\copyright 2025 David Foxcroft and Danielle Navarro

This work is licensed under an Attribution-ShareAlike 4.0 International (CC BY-SA 4.0).

This license allows you to copy and redistribute, transform, and build upon the material for any purpose, even commercially. providing attribution is made to the authors (but not in any way that suggests that they endorse you or your use of the work). Attribution should include the following information:

Danielle Navarro and David Foxcroft, \textit{Learning statistics with jamovi: A tutorial for beginners in statistical analysis}. Cambridge, UK: Open Book Publishers, 2025, \url{https://doi.org/10.11647/OBP.0333}

Further details about CC BY-SA licenses are available at \\ \url{https://creativecommons.org/licenses/by-sa/4.0/}

All external links were active at the time of publication unless otherwise stated and have been archived via the Internet Archive Wayback Machine at \\ \url{https://archive.org/web}

Digital material and resources associated with this volume are available at \\ \url{https://doi.org/10.11647/OBP.0333\#resources}

ISBN Paperback: 978-1-80064-937-8

ISBN Hardback: 978-1-80064-938-5

ISBN Digital (PDF): 978-1-80064-939-2

DOI: 10.11647/OBP.0333


\vfill
\hspace{0pt}
\pagebreak

\hspace{0pt}
\vfill

This textbook covers the contents of an introductory statistics class, as typically taught to undergraduate psychology, health or social science students. The book covers how to get started in jamovi as well as giving an introduction to data manipulation. From a statistical perspective, the book discusses descriptive statistics and graphing first, followed by chapters on probability theory, sampling and estimation, and null hypothesis testing. After introducing the theory, the book covers the analysis of contingency tables, correlation, \textit{t}-tests, regression, ANOVA and factor analysis. Bayesian statistics are touched on at the end of the book.

Data sets used in the book are freely available for use in jamovi. All the data files you need can be accessed within jamovi via an add-on module in the jamovi library. Or you can download the files from \url{https://www.learnstatswithjamovi.com}.


Citation: Danielle Navarro and David Foxcroft, \textit{Learning statistics with jamovi: A tutorial for beginners in statistical analysis}. Cambridge, UK: Open Book Publishers, 2025, \url{https://doi.org/10.11647/OBP.0333}

\vfill
\hspace{0pt}

\ifdefined\Shaded\renewenvironment{Shaded}{\begin{tcolorbox}[borderline west={3pt}{0pt}{shadecolor}, boxrule=0pt, enhanced, breakable, frame hidden, sharp corners, interior hidden]}{\end{tcolorbox}}\fi

\renewcommand*\contentsname{Table of contents}
{
\hypersetup{linkcolor=}
\setcounter{tocdepth}{2}
\tableofcontents
}
\mainmatter
\bookmarksetup{startatroot}

\hypertarget{preface}{%
\chapter*{Preface}\label{preface}}
\addcontentsline{toc}{chapter}{Preface}

\markboth{Preface}{Preface}

This book is an adaptation of DJ Navarro (2018). Learning statistics
with R: A tutorial for psychology students and other beginners. (Version
0.6). \url{https://learningstatisticswithr.com/}.

The jamovi version of this book was first released in 2018, as version
0.65. Versions 0.70 and 0.75 were released in subsequent years with
corrections and additions; details of the changes in earlier versions of
the book can be found in the preface to version 0.75:
\url{https://github.com/user-attachments/files/18124061/learning-statistics-with-jamovi-0.75.pdf}.
In that time, many people have contacted us asking for a hard copy
version of the book. To achieve this, and to preserve the open source
attributes of the book and materials, we have worked with
\href{https://www.openbookpublishers.com/books/10.11647/obp.0333}{Open
Book Publishers} in Cambridge, UK, to release this updated version. Open
Book Publishers are the leading independent open access publisher of
academic research in the Humanities and Social Sciences in the UK. They
are award-winning, not-for-profit, run by scholars, and committed to
making high-quality research freely available to readers around the
world.

If you spot any mistakes, or have any suggestions, please do let us know
by raising an issue at
\url{https://github.com/davidfoxcroft/lsj-book/issues}.

\emph{David Foxcroft\\
January 1st, 2025}

\part{Beginnings}

\hypertarget{why-do-we-learn-statistics}{%
\chapter{Why do we learn statistics}\label{why-do-we-learn-statistics}}

\placetextbox{0.26}{0.06}{\scriptsize{©2025 D. Foxcroft and D. Navarro,}}
\placetextbox{0.205}{0.05}{\scriptsize{CC BY-SA 4.0}}
\placetextbox{0.70}{0.06}{\scriptsize{\url{https://doi.org/10.11647/OBP.0333/01}}}

\[ \]

\begin{quote}
\emph{Thou shalt not answer questionnaires}\\
\emph{Or quizzes upon World Affairs,}\\
\emph{Nor with compliance}\\
\emph{Take any test. Thou shalt not sit}\\
\emph{With statisticians nor commit}\\
\emph{A social science}\\
-- W.H. Auden\footnote{The quote comes from Auden's 1946 poem
  \emph{Under Which Lyre: A Reactionary Tract for the Times}, delivered
  as part of a commencement address at Harvard University. The history
  of the poem is kind of interesting, see Adam Kirsch's analysis in the
  \emph{Harvard Magazine},
  \url{https://www.harvardmagazine.com/2007/11/a-poets-warning.html}}
\end{quote}

\hypertarget{on-the-psychology-of-statistics}{%
\section{On the psychology of
statistics}\label{on-the-psychology-of-statistics}}

To the surprise of many students, statistics is a fairly significant
part of a psychological education. To the surprise of no-one, statistics
is very rarely the \emph{favourite} part of one's psychological
education. After all, if you really loved the idea of doing statistics,
you'd probably be enrolled in a statistics class right now, not a
psychology class. So, not surprisingly, there's a pretty large
proportion of the student base that isn't happy about the fact that
psychology has so much statistics in it. In view of this, I thought that
the right place to start might be to answer some of the more common
questions that people have about stats.

A big part of this issue at hand relates to the very idea of statistics.
What is it? What's it there for? And why are scientists so bloody
obsessed with it? These are all good questions, when you think about it.
So let's start with the last one. As a group, scientists seem to be
bizarrely fixated on running statistical tests on everything. In fact,
we use statistics so often that we sometimes forget to explain to people
why we do. It's a kind of article of faith among scientists -- and
especially social scientists -- that your findings can't be trusted
until you've done some stats. Undergraduate students might be forgiven
for thinking that we're all completely mad, because no-one takes the
time to answer one very simple question:

\begin{quote}
\emph{Why do you do statistics? Why don't scientists just use}
\textbf{common sense}?
\end{quote}

It's a naive question in some ways, but most good questions are. There's
a lot of good answers to it,\footnote{Including the suggestion that
  common sense is in short supply among scientists.} but for my money,
the best answer is a really simple one: we don't trust ourselves enough.
We worry that we're human, and susceptible to all of the biases,
temptations and frailties that humans suffer from. Much of statistics is
basically a safeguard. Using ``common sense'' to evaluate evidence means
trusting gut instincts, relying on verbal arguments and on using the raw
power of human reason to come up with the right answer. Most scientists
don't think this approach is likely to work.

In fact, come to think of it, this sounds a lot like a psychological
question to me, and since I do work in a psychology department, it seems
like a good idea to dig a little deeper here. Is it really plausible to
think that this ``common sense'' approach is very trustworthy? Verbal
arguments have to be constructed in language, and all languages have
biases -- some things are harder to say than others, and not necessarily
because they're false (e.g., quantum electrodynamics is a good theory,
but hard to explain in words). The instincts of our ``gut'' aren't
designed to solve scientific problems, they're designed to handle
day-to-day inferences -- and given that biological evolution is slower
than cultural change, we should say that they're designed to solve the
day-to-day problems for a \emph{different world} than the one we live
in. Most fundamentally, reasoning sensibly requires people to engage in
``induction'', making wise guesses and going beyond the immediate
evidence of the senses to make generalisations about the world. If you
think that you can do that without being influenced by various
distractors, well, I have a bridge in London I'd like to sell you. Heck,
as the next section shows, we can't even solve ``deductive'' problems
(ones where no guessing is required) without being influenced by our
pre-existing biases.

\hypertarget{the-curse-of-belief-bias}{%
\subsection{The curse of belief bias}\label{the-curse-of-belief-bias}}

People are mostly pretty smart. We're smarter than the other species
that we share the planet with (though many people might disagree). Our
minds are quite amazing things, and we seem to be capable of the most
incredible feats of thought and reason. That doesn't make us perfect
though. And among the many things that psychologists have shown over the
years is that we really do find it hard to be neutral, to evaluate
evidence impartially and without being swayed by pre-existing biases. A
good example of this is the \textbf{belief bias effect} in logical
reasoning: if you ask people to decide whether a particular argument is
logically valid (i.e., the conclusion would be true if the premises were
true), we tend to be influenced by the believability of the conclusion,
even when we shouldn't. For instance, here's a valid argument where the
conclusion is believable:

\begin{quote}
All cigarettes are expensive (Premise 1)\\
Some addictive things are inexpensive (Premise 2)\\
Therefore, some addictive things are not cigarettes (Conclusion)
\end{quote}

And here's a valid argument where the conclusion is not believable:

\begin{quote}
All addictive things are expensive (Premise 1)\\
Some cigarettes are inexpensive (Premise 2)\\
Therefore, some cigarettes are not addictive (Conclusion)
\end{quote}

The logical \emph{structure} of argument \#2 is identical to the
structure of argument \#1, and they're both valid. However, in the
second argument, there are good reasons to think that premise 1 is
incorrect, and as a result it's probably the case that the conclusion is
also incorrect. But that's entirely irrelevant to the topic at hand; an
argument is deductively valid if the conclusion is a logical consequence
of the premises. That is, a valid argument doesn't have to involve true
statements.

On the other hand, here's an invalid argument that has a believable
conclusion:

\begin{quote}
All addictive things are expensive (Premise 1)\\
Some cigarettes are inexpensive (Premise 2)\\
Therefore, some addictive things are not cigarettes (Conclusion)
\end{quote}

And finally, an invalid argument with an unbelievable conclusion:

\begin{quote}
All cigarettes are expensive (Premise 1)\\
Some addictive things are inexpensive (Premise 2)\\
Therefore, some cigarettes are not addictive (Conclusion)
\end{quote}

Now, suppose that people really are perfectly able to set aside their
pre-existing biases about what is true and what isn't, and purely
evaluate an argument on its logical merits. We'd expect 100\% of people
to say that the valid arguments are valid, and 0\% of people to say that
the invalid arguments are valid. So if you ran an experiment looking at
this, you'd expect to see data as in Table~\ref{tbl-tab1-1}.

\hypertarget{tbl-tab1-1}{}
 
  \providecommand{\huxb}[2]{\arrayrulecolor[RGB]{#1}\global\arrayrulewidth=#2pt}
  \providecommand{\huxvb}[2]{\color[RGB]{#1}\vrule width #2pt}
  \providecommand{\huxtpad}[1]{\rule{0pt}{#1}}
  \providecommand{\huxbpad}[1]{\rule[-#1]{0pt}{#1}}

\begin{table}[ht]
\caption{\label{tbl-tab1-1}Validity of arguments }\tabularnewline

\begin{centerbox}
\begin{threeparttable}
\setlength{\tabcolsep}{0pt}
\begin{tabularx}{0.9\textwidth}{p{0.3\textwidth} p{0.3\textwidth} p{0.3\textwidth}}


\hhline{>{\huxb{0, 0, 0}{0.4}}->{\huxb{0, 0, 0}{0.4}}->{\huxb{0, 0, 0}{0.4}}-}
\arrayrulecolor{black}

\multicolumn{1}{!{\huxvb{0, 0, 0}{0}}p{0.3\textwidth}!{\huxvb{0, 0, 0}{0}}}{\hspace{0pt}\parbox[b]{0.3\textwidth-0pt-12pt}{\huxtpad{2pt + 1em}\centering \textbf{}\huxbpad{2pt}}} &
\multicolumn{1}{p{0.3\textwidth}!{\huxvb{0, 0, 0}{0}}}{\hspace{12pt}\parbox[b]{0.3\textwidth-12pt-12pt}{\huxtpad{2pt + 1em}\centering \textbf{conclusion feels true}\huxbpad{2pt}}} &
\multicolumn{1}{p{0.3\textwidth}!{\huxvb{0, 0, 0}{0}}}{\hspace{12pt}\parbox[b]{0.3\textwidth-12pt-0pt}{\huxtpad{2pt + 1em}\centering \textbf{conclusion feels false}\huxbpad{2pt}}} \tabularnewline[-0.5pt]


\hhline{>{\huxb{0, 0, 0}{0.4}}->{\huxb{0, 0, 0}{0.4}}->{\huxb{0, 0, 0}{0.4}}-}
\arrayrulecolor{black}

\multicolumn{1}{!{\huxvb{0, 0, 0}{0}}p{0.3\textwidth}!{\huxvb{0, 0, 0}{0}}}{\hspace{0pt}\parbox[b]{0.3\textwidth-0pt-12pt}{\huxtpad{2pt + 1em}\centering \textbf{argument is valid}\huxbpad{2pt}}} &
\multicolumn{1}{p{0.3\textwidth}!{\huxvb{0, 0, 0}{0}}}{\hspace{12pt}\parbox[b]{0.3\textwidth-12pt-12pt}{\huxtpad{2pt + 1em}\centering 100\(\%\) say \(\text{``}\)valid\(\text{''}\)\huxbpad{2pt}}} &
\multicolumn{1}{p{0.3\textwidth}!{\huxvb{0, 0, 0}{0}}}{\hspace{12pt}\parbox[b]{0.3\textwidth-12pt-0pt}{\huxtpad{2pt + 1em}\centering 100\(\%\) say \(\text{``}\)valid\(\text{''}\)\huxbpad{2pt}}} \tabularnewline[-0.5pt]


\hhline{}
\arrayrulecolor{black}

\multicolumn{1}{!{\huxvb{0, 0, 0}{0}}p{0.3\textwidth}!{\huxvb{0, 0, 0}{0}}}{\hspace{0pt}\parbox[b]{0.3\textwidth-0pt-12pt}{\huxtpad{2pt + 1em}\centering \textbf{argument is invalid}\huxbpad{2pt}}} &
\multicolumn{1}{p{0.3\textwidth}!{\huxvb{0, 0, 0}{0}}}{\hspace{12pt}\parbox[b]{0.3\textwidth-12pt-12pt}{\huxtpad{2pt + 1em}\centering 0\(\%\) say \(\text{``}\)valid\(\text{''}\)\huxbpad{2pt}}} &
\multicolumn{1}{p{0.3\textwidth}!{\huxvb{0, 0, 0}{0}}}{\hspace{12pt}\parbox[b]{0.3\textwidth-12pt-0pt}{\huxtpad{2pt + 1em}\centering 0\(\%\) say \(\text{``}\)valid\(\text{''}\)\huxbpad{2pt}}} \tabularnewline[-0.5pt]


\hhline{>{\huxb{0, 0, 0}{0.4}}->{\huxb{0, 0, 0}{0.4}}->{\huxb{0, 0, 0}{0.4}}-}
\arrayrulecolor{black}
\end{tabularx} 

\end{threeparttable}\par\end{centerbox}

\end{table}
 

If the psychological data looked like this (or even a good approximation
to this), we might feel safe in just trusting our gut instincts. That
is, it'd be perfectly okay just to let scientists evaluate data based on
their common sense, and not bother with all this murky statistics stuff.
However, you guys have taken psych classes, and by now you probably know
where this is going.

In a classic study, Evans et al. (1983) ran an experiment looking at
exactly this. What they found is that when pre-existing biases (i.e.,
beliefs) were in agreement with the structure of the data, everything
went the way you'd hope (Table~\ref{tbl-tab1-2}).

\hypertarget{tbl-tab1-2}{}
 
  \providecommand{\huxb}[2]{\arrayrulecolor[RGB]{#1}\global\arrayrulewidth=#2pt}
  \providecommand{\huxvb}[2]{\color[RGB]{#1}\vrule width #2pt}
  \providecommand{\huxtpad}[1]{\rule{0pt}{#1}}
  \providecommand{\huxbpad}[1]{\rule[-#1]{0pt}{#1}}

\begin{table}[ht]
\caption{\label{tbl-tab1-2}Pre-existing biases and argument validity }\tabularnewline

\begin{centerbox}
\begin{threeparttable}
\setlength{\tabcolsep}{0pt}
\begin{tabularx}{0.9\textwidth}{p{0.3\textwidth} p{0.3\textwidth} p{0.3\textwidth}}


\hhline{>{\huxb{0, 0, 0}{0.4}}->{\huxb{0, 0, 0}{0.4}}->{\huxb{0, 0, 0}{0.4}}-}
\arrayrulecolor{black}

\multicolumn{1}{!{\huxvb{0, 0, 0}{0}}p{0.3\textwidth}!{\huxvb{0, 0, 0}{0}}}{\hspace{0pt}\parbox[b]{0.3\textwidth-0pt-12pt}{\huxtpad{2pt + 1em}\centering \textbf{}\huxbpad{2pt}}} &
\multicolumn{1}{p{0.3\textwidth}!{\huxvb{0, 0, 0}{0}}}{\hspace{12pt}\parbox[b]{0.3\textwidth-12pt-12pt}{\huxtpad{2pt + 1em}\centering \textbf{conclusion feels true}\huxbpad{2pt}}} &
\multicolumn{1}{p{0.3\textwidth}!{\huxvb{0, 0, 0}{0}}}{\hspace{12pt}\parbox[b]{0.3\textwidth-12pt-0pt}{\huxtpad{2pt + 1em}\centering \textbf{conclusion feels false}\huxbpad{2pt}}} \tabularnewline[-0.5pt]


\hhline{>{\huxb{0, 0, 0}{0.4}}->{\huxb{0, 0, 0}{0.4}}->{\huxb{0, 0, 0}{0.4}}-}
\arrayrulecolor{black}

\multicolumn{1}{!{\huxvb{0, 0, 0}{0}}p{0.3\textwidth}!{\huxvb{0, 0, 0}{0}}}{\hspace{0pt}\parbox[b]{0.3\textwidth-0pt-12pt}{\huxtpad{2pt + 1em}\centering \textbf{argument is valid}\huxbpad{2pt}}} &
\multicolumn{1}{p{0.3\textwidth}!{\huxvb{0, 0, 0}{0}}}{\hspace{12pt}\parbox[b]{0.3\textwidth-12pt-12pt}{\huxtpad{2pt + 1em}\centering 92\(\%\) say \(\text{``}\)valid\(\text{''}\)\huxbpad{2pt}}} &
\multicolumn{1}{p{0.3\textwidth}!{\huxvb{0, 0, 0}{0}}}{\hspace{12pt}\parbox[b]{0.3\textwidth-12pt-0pt}{\huxtpad{2pt + 1em}\centering \huxbpad{2pt}}} \tabularnewline[-0.5pt]


\hhline{}
\arrayrulecolor{black}

\multicolumn{1}{!{\huxvb{0, 0, 0}{0}}p{0.3\textwidth}!{\huxvb{0, 0, 0}{0}}}{\hspace{0pt}\parbox[b]{0.3\textwidth-0pt-12pt}{\huxtpad{2pt + 1em}\centering \textbf{argument is invalid}\huxbpad{2pt}}} &
\multicolumn{1}{p{0.3\textwidth}!{\huxvb{0, 0, 0}{0}}}{\hspace{12pt}\parbox[b]{0.3\textwidth-12pt-12pt}{\huxtpad{2pt + 1em}\centering \huxbpad{2pt}}} &
\multicolumn{1}{p{0.3\textwidth}!{\huxvb{0, 0, 0}{0}}}{\hspace{12pt}\parbox[b]{0.3\textwidth-12pt-0pt}{\huxtpad{2pt + 1em}\centering 8\(\%\) say \(\text{``}\)valid\(\text{''}\)\huxbpad{2pt}}} \tabularnewline[-0.5pt]


\hhline{>{\huxb{0, 0, 0}{0.4}}->{\huxb{0, 0, 0}{0.4}}->{\huxb{0, 0, 0}{0.4}}-}
\arrayrulecolor{black}
\end{tabularx} 

\end{threeparttable}\par\end{centerbox}

\end{table}
 

Not perfect, but that's pretty good. But look what happens when our
intuitive feelings about the truth of the conclusion run against the
logical structure of the argument, see (Table~\ref{tbl-tab1-3}).

Oh dear, that's not as good. Apparently, when people are presented with
a strong argument that contradicts our pre-existing beliefs, we find it
pretty hard to even perceive it to be a strong argument (people only did
so 46\% of the time). Even worse, when people are presented with a weak
argument that agrees with our pre-existing biases, almost no-one can see
that the argument is weak (people got that one wrong 92\% of the
time!).\footnote{In my more cynical moments I feel like this fact alone
  explains 95\% of what I read on the internet.}

\hypertarget{tbl-tab1-3}{}
 
  \providecommand{\huxb}[2]{\arrayrulecolor[RGB]{#1}\global\arrayrulewidth=#2pt}
  \providecommand{\huxvb}[2]{\color[RGB]{#1}\vrule width #2pt}
  \providecommand{\huxtpad}[1]{\rule{0pt}{#1}}
  \providecommand{\huxbpad}[1]{\rule[-#1]{0pt}{#1}}

\begin{table}[ht]
\caption{\label{tbl-tab1-3}Intuition and argument validity }\tabularnewline

\begin{centerbox}
\begin{threeparttable}
\setlength{\tabcolsep}{0pt}
\begin{tabularx}{0.9\textwidth}{p{0.3\textwidth} p{0.3\textwidth} p{0.3\textwidth}}


\hhline{>{\huxb{0, 0, 0}{0.4}}->{\huxb{0, 0, 0}{0.4}}->{\huxb{0, 0, 0}{0.4}}-}
\arrayrulecolor{black}

\multicolumn{1}{!{\huxvb{0, 0, 0}{0}}p{0.3\textwidth}!{\huxvb{0, 0, 0}{0}}}{\hspace{0pt}\parbox[b]{0.3\textwidth-0pt-12pt}{\huxtpad{2pt + 1em}\centering \textbf{}\huxbpad{2pt}}} &
\multicolumn{1}{p{0.3\textwidth}!{\huxvb{0, 0, 0}{0}}}{\hspace{12pt}\parbox[b]{0.3\textwidth-12pt-12pt}{\huxtpad{2pt + 1em}\centering \textbf{conclusion feels true}\huxbpad{2pt}}} &
\multicolumn{1}{p{0.3\textwidth}!{\huxvb{0, 0, 0}{0}}}{\hspace{12pt}\parbox[b]{0.3\textwidth-12pt-0pt}{\huxtpad{2pt + 1em}\centering \textbf{conclusion feels false}\huxbpad{2pt}}} \tabularnewline[-0.5pt]


\hhline{>{\huxb{0, 0, 0}{0.4}}->{\huxb{0, 0, 0}{0.4}}->{\huxb{0, 0, 0}{0.4}}-}
\arrayrulecolor{black}

\multicolumn{1}{!{\huxvb{0, 0, 0}{0}}p{0.3\textwidth}!{\huxvb{0, 0, 0}{0}}}{\hspace{0pt}\parbox[b]{0.3\textwidth-0pt-12pt}{\huxtpad{2pt + 1em}\centering \textbf{argument is valid}\huxbpad{2pt}}} &
\multicolumn{1}{p{0.3\textwidth}!{\huxvb{0, 0, 0}{0}}}{\hspace{12pt}\parbox[b]{0.3\textwidth-12pt-12pt}{\huxtpad{2pt + 1em}\centering 92\(\%\) say \(\text{``}\)valid\(\text{''}\)\huxbpad{2pt}}} &
\multicolumn{1}{p{0.3\textwidth}!{\huxvb{0, 0, 0}{0}}}{\hspace{12pt}\parbox[b]{0.3\textwidth-12pt-0pt}{\huxtpad{2pt + 1em}\centering 46\(\%\) say \(\text{``}\)valid\(\text{''}\)\huxbpad{2pt}}} \tabularnewline[-0.5pt]


\hhline{}
\arrayrulecolor{black}

\multicolumn{1}{!{\huxvb{0, 0, 0}{0}}p{0.3\textwidth}!{\huxvb{0, 0, 0}{0}}}{\hspace{0pt}\parbox[b]{0.3\textwidth-0pt-12pt}{\huxtpad{2pt + 1em}\centering \textbf{argument is invalid}\huxbpad{2pt}}} &
\multicolumn{1}{p{0.3\textwidth}!{\huxvb{0, 0, 0}{0}}}{\hspace{12pt}\parbox[b]{0.3\textwidth-12pt-12pt}{\huxtpad{2pt + 1em}\centering 92\(\%\) say \(\text{``}\)valid\(\text{''}\)\huxbpad{2pt}}} &
\multicolumn{1}{p{0.3\textwidth}!{\huxvb{0, 0, 0}{0}}}{\hspace{12pt}\parbox[b]{0.3\textwidth-12pt-0pt}{\huxtpad{2pt + 1em}\centering 8\(\%\) say \(\text{``}\)valid\(\text{''}\)\huxbpad{2pt}}} \tabularnewline[-0.5pt]


\hhline{>{\huxb{0, 0, 0}{0.4}}->{\huxb{0, 0, 0}{0.4}}->{\huxb{0, 0, 0}{0.4}}-}
\arrayrulecolor{black}
\end{tabularx} 

\end{threeparttable}\par\end{centerbox}

\end{table}
 

If you think about it, it's not as if these data are horribly damning.
Overall, people did do better than chance at compensating for their
prior biases, since about 60\% of people's judgements were correct
(you'd expect 50\% by chance). Even so, if you were a professional
``evaluator of evidence'', and someone came along and offered you a
magic tool that improves your chances of making the right decision from
60\% to (say) 95\%, you'd probably jump at it, right? Of course you
would. Thankfully, we actually do have a tool that can do this. But it's
not magic, it's statistics. So that's reason \#1 why scientists love
statistics. It's just too easy for us to ``believe what we want to
believe''. So instead, if we want to ``believe in the data'', we're
going to need a bit of help to keep our personal biases under control.
That's what statistics does, it helps keep us honest.

\hypertarget{the-cautionary-tale-of-simpsons-paradox}{%
\section{The cautionary tale of Simpson's
paradox}\label{the-cautionary-tale-of-simpsons-paradox}}

The following is a true story (I think!). In 1973, the University of
California, Berkeley had some worries about the admissions of students
into their postgraduate courses. Specifically, the thing that caused the
problem was the gender breakdown of their admissions
(Table~\ref{tbl-tab1-4}).

\hypertarget{tbl-tab1-4}{}
 
  \providecommand{\huxb}[2]{\arrayrulecolor[RGB]{#1}\global\arrayrulewidth=#2pt}
  \providecommand{\huxvb}[2]{\color[RGB]{#1}\vrule width #2pt}
  \providecommand{\huxtpad}[1]{\rule{0pt}{#1}}
  \providecommand{\huxbpad}[1]{\rule[-#1]{0pt}{#1}}

\begin{table}[ht]
\caption{\label{tbl-tab1-4}Berkeley students by gender }\tabularnewline

\begin{centerbox}
\begin{threeparttable}
\setlength{\tabcolsep}{0pt}
\begin{tabularx}{0.9\textwidth}{p{0.3\textwidth} p{0.3\textwidth} p{0.3\textwidth}}


\hhline{>{\huxb{0, 0, 0}{0.4}}->{\huxb{0, 0, 0}{0.4}}->{\huxb{0, 0, 0}{0.4}}-}
\arrayrulecolor{black}

\multicolumn{1}{!{\huxvb{0, 0, 0}{0}}p{0.3\textwidth}!{\huxvb{0, 0, 0}{0}}}{\hspace{0pt}\parbox[b]{0.3\textwidth-0pt-12pt}{\huxtpad{2pt + 1em}\centering \textbf{}\huxbpad{2pt}}} &
\multicolumn{1}{p{0.3\textwidth}!{\huxvb{0, 0, 0}{0}}}{\hspace{12pt}\parbox[b]{0.3\textwidth-12pt-12pt}{\huxtpad{2pt + 1em}\centering \textbf{Number of applicants}\huxbpad{2pt}}} &
\multicolumn{1}{p{0.3\textwidth}!{\huxvb{0, 0, 0}{0}}}{\hspace{12pt}\parbox[b]{0.3\textwidth-12pt-0pt}{\huxtpad{2pt + 1em}\centering \textbf{Percent admitted}\huxbpad{2pt}}} \tabularnewline[-0.5pt]


\hhline{>{\huxb{0, 0, 0}{0.4}}->{\huxb{0, 0, 0}{0.4}}->{\huxb{0, 0, 0}{0.4}}-}
\arrayrulecolor{black}

\multicolumn{1}{!{\huxvb{0, 0, 0}{0}}p{0.3\textwidth}!{\huxvb{0, 0, 0}{0}}}{\hspace{0pt}\parbox[b]{0.3\textwidth-0pt-12pt}{\huxtpad{2pt + 1em}\centering Males\huxbpad{2pt}}} &
\multicolumn{1}{p{0.3\textwidth}!{\huxvb{0, 0, 0}{0}}}{\hspace{12pt}\parbox[b]{0.3\textwidth-12pt-12pt}{\huxtpad{2pt + 1em}\centering 8442\huxbpad{2pt}}} &
\multicolumn{1}{p{0.3\textwidth}!{\huxvb{0, 0, 0}{0}}}{\hspace{12pt}\parbox[b]{0.3\textwidth-12pt-0pt}{\huxtpad{2pt + 1em}\centering 44\(\%\)\huxbpad{2pt}}} \tabularnewline[-0.5pt]


\hhline{}
\arrayrulecolor{black}

\multicolumn{1}{!{\huxvb{0, 0, 0}{0}}p{0.3\textwidth}!{\huxvb{0, 0, 0}{0}}}{\hspace{0pt}\parbox[b]{0.3\textwidth-0pt-12pt}{\huxtpad{2pt + 1em}\centering Females\huxbpad{2pt}}} &
\multicolumn{1}{p{0.3\textwidth}!{\huxvb{0, 0, 0}{0}}}{\hspace{12pt}\parbox[b]{0.3\textwidth-12pt-12pt}{\huxtpad{2pt + 1em}\centering 4321\huxbpad{2pt}}} &
\multicolumn{1}{p{0.3\textwidth}!{\huxvb{0, 0, 0}{0}}}{\hspace{12pt}\parbox[b]{0.3\textwidth-12pt-0pt}{\huxtpad{2pt + 1em}\centering 35\(\%\)\huxbpad{2pt}}} \tabularnewline[-0.5pt]


\hhline{>{\huxb{0, 0, 0}{0.4}}->{\huxb{0, 0, 0}{0.4}}->{\huxb{0, 0, 0}{0.4}}-}
\arrayrulecolor{black}
\end{tabularx} 

\end{threeparttable}\par\end{centerbox}

\end{table}
 

Given this, they were worried about being sued!\footnote{Earlier
  versions of these notes incorrectly suggested that they actually were
  sued. But that's not true. There's a nice commentary on this by Alex
  Reinhart here:
  \url{https://www.refsmmat.com/posts/2016-05-08-simpsons-paradox-berkeley.html}
  A big thank you to Wilfried Van Hirtum for pointing this out to me.}
Given that there were nearly 13,000 applicants, a difference of 9\% in
admission rates between males and females is just way too big to be a
coincidence. Pretty compelling data, right? And if I were to say to you
that these data \emph{actually} reflect a weak bias in favour of women
(sort of!), you'd probably think that I was either crazy or sexist.

Oddly, it's actually sort of true. When people started looking more
carefully at the admissions data they told a rather different story
(Bickel et al., 1975). Specifically, when they looked at it on a
department by department basis, it turned out that most of the
departments actually had a slightly \emph{higher} success rate for
female applicants than for male applicants. Table~\ref{tbl-tab1-5} shows
the admission figures for the six largest departments (with the names of
the departments removed for privacy reasons):

\hypertarget{tbl-tab1-5}{}
 
  \providecommand{\huxb}[2]{\arrayrulecolor[RGB]{#1}\global\arrayrulewidth=#2pt}
  \providecommand{\huxvb}[2]{\color[RGB]{#1}\vrule width #2pt}
  \providecommand{\huxtpad}[1]{\rule{0pt}{#1}}
  \providecommand{\huxbpad}[1]{\rule[-#1]{0pt}{#1}}

\begin{table}[ht]
\caption{\label{tbl-tab1-5}Berkeley students by gender for six largest departments }\tabularnewline

\begin{centerbox}
\begin{threeparttable}
\setlength{\tabcolsep}{0pt}
\begin{tabularx}{0.9\textwidth}{p{0.18\textwidth} p{0.18\textwidth} p{0.18\textwidth} p{0.18\textwidth} p{0.18\textwidth}}


\hhline{>{\huxb{0, 0, 0}{0.4}}->{\huxb{0, 0, 0}{0.4}}->{\huxb{0, 0, 0}{0.4}}->{\huxb{0, 0, 0}{0.4}}->{\huxb{0, 0, 0}{0.4}}-}
\arrayrulecolor{black}

\multicolumn{1}{!{\huxvb{0, 0, 0}{0}}p{0.18\textwidth}!{\huxvb{0, 0, 0}{0}}}{\hspace{0pt}\parbox[b]{0.18\textwidth-0pt-12pt}{\huxtpad{2pt + 1em}\centering \textbf{}\huxbpad{2pt}}} &
\multicolumn{2}{p{0.36\textwidth+2\tabcolsep}!{\huxvb{0, 0, 0}{0}}}{\hspace{12pt}\parbox[b]{0.36\textwidth+2\tabcolsep-12pt-12pt}{\huxtpad{2pt + 1em}\centering \textbf{Males}\huxbpad{2pt}}} &
\multicolumn{2}{p{0.36\textwidth+2\tabcolsep}!{\huxvb{0, 0, 0}{0}}}{\hspace{12pt}\parbox[b]{0.36\textwidth+2\tabcolsep-12pt-12pt}{\huxtpad{2pt + 1em}\centering \textbf{Females}\huxbpad{2pt}}} \tabularnewline[-0.5pt]


\hhline{>{\huxb{0, 0, 0}{0.4}}->{\huxb{0, 0, 0}{0.4}}->{\huxb{0, 0, 0}{0.4}}->{\huxb{0, 0, 0}{0.4}}->{\huxb{0, 0, 0}{0.4}}-}
\arrayrulecolor{black}

\multicolumn{1}{!{\huxvb{0, 0, 0}{0}}p{0.18\textwidth}!{\huxvb{0, 0, 0}{0}}}{\hspace{0pt}\parbox[b]{0.18\textwidth-0pt-12pt}{\huxtpad{2pt + 1em}\centering Department\huxbpad{2pt}}} &
\multicolumn{1}{p{0.18\textwidth}!{\huxvb{0, 0, 0}{0}}}{\hspace{12pt}\parbox[b]{0.18\textwidth-12pt-12pt}{\huxtpad{2pt + 1em}\centering Applicants\huxbpad{2pt}}} &
\multicolumn{1}{p{0.18\textwidth}!{\huxvb{0, 0, 0}{0}}}{\hspace{12pt}\parbox[b]{0.18\textwidth-12pt-12pt}{\huxtpad{2pt + 1em}\centering Percent admitted\huxbpad{2pt}}} &
\multicolumn{1}{p{0.18\textwidth}!{\huxvb{0, 0, 0}{0}}}{\hspace{12pt}\parbox[b]{0.18\textwidth-12pt-12pt}{\huxtpad{2pt + 1em}\centering Applicants\huxbpad{2pt}}} &
\multicolumn{1}{p{0.18\textwidth}!{\huxvb{0, 0, 0}{0}}}{\hspace{12pt}\parbox[b]{0.18\textwidth-12pt-0pt}{\huxtpad{2pt + 1em}\centering Percent admitted\huxbpad{2pt}}} \tabularnewline[-0.5pt]


\hhline{}
\arrayrulecolor{black}

\multicolumn{1}{!{\huxvb{0, 0, 0}{0}}p{0.18\textwidth}!{\huxvb{0, 0, 0}{0}}}{\hspace{0pt}\parbox[b]{0.18\textwidth-0pt-12pt}{\huxtpad{2pt + 1em}\centering A\huxbpad{2pt}}} &
\multicolumn{1}{p{0.18\textwidth}!{\huxvb{0, 0, 0}{0}}}{\hspace{12pt}\parbox[b]{0.18\textwidth-12pt-12pt}{\huxtpad{2pt + 1em}\centering 825\huxbpad{2pt}}} &
\multicolumn{1}{p{0.18\textwidth}!{\huxvb{0, 0, 0}{0}}}{\hspace{12pt}\parbox[b]{0.18\textwidth-12pt-12pt}{\huxtpad{2pt + 1em}\centering 62\(\%\)\huxbpad{2pt}}} &
\multicolumn{1}{p{0.18\textwidth}!{\huxvb{0, 0, 0}{0}}}{\hspace{12pt}\parbox[b]{0.18\textwidth-12pt-12pt}{\huxtpad{2pt + 1em}\centering 108\huxbpad{2pt}}} &
\multicolumn{1}{p{0.18\textwidth}!{\huxvb{0, 0, 0}{0}}}{\hspace{12pt}\parbox[b]{0.18\textwidth-12pt-0pt}{\huxtpad{2pt + 1em}\centering 82\(\%\)\huxbpad{2pt}}} \tabularnewline[-0.5pt]


\hhline{}
\arrayrulecolor{black}

\multicolumn{1}{!{\huxvb{0, 0, 0}{0}}p{0.18\textwidth}!{\huxvb{0, 0, 0}{0}}}{\hspace{0pt}\parbox[b]{0.18\textwidth-0pt-12pt}{\huxtpad{2pt + 1em}\centering B\huxbpad{2pt}}} &
\multicolumn{1}{p{0.18\textwidth}!{\huxvb{0, 0, 0}{0}}}{\hspace{12pt}\parbox[b]{0.18\textwidth-12pt-12pt}{\huxtpad{2pt + 1em}\centering 560\huxbpad{2pt}}} &
\multicolumn{1}{p{0.18\textwidth}!{\huxvb{0, 0, 0}{0}}}{\hspace{12pt}\parbox[b]{0.18\textwidth-12pt-12pt}{\huxtpad{2pt + 1em}\centering 63\(\%\)\huxbpad{2pt}}} &
\multicolumn{1}{p{0.18\textwidth}!{\huxvb{0, 0, 0}{0}}}{\hspace{12pt}\parbox[b]{0.18\textwidth-12pt-12pt}{\huxtpad{2pt + 1em}\centering 25\huxbpad{2pt}}} &
\multicolumn{1}{p{0.18\textwidth}!{\huxvb{0, 0, 0}{0}}}{\hspace{12pt}\parbox[b]{0.18\textwidth-12pt-0pt}{\huxtpad{2pt + 1em}\centering 68\(\%\)\huxbpad{2pt}}} \tabularnewline[-0.5pt]


\hhline{}
\arrayrulecolor{black}

\multicolumn{1}{!{\huxvb{0, 0, 0}{0}}p{0.18\textwidth}!{\huxvb{0, 0, 0}{0}}}{\hspace{0pt}\parbox[b]{0.18\textwidth-0pt-12pt}{\huxtpad{2pt + 1em}\centering C\huxbpad{2pt}}} &
\multicolumn{1}{p{0.18\textwidth}!{\huxvb{0, 0, 0}{0}}}{\hspace{12pt}\parbox[b]{0.18\textwidth-12pt-12pt}{\huxtpad{2pt + 1em}\centering 325\huxbpad{2pt}}} &
\multicolumn{1}{p{0.18\textwidth}!{\huxvb{0, 0, 0}{0}}}{\hspace{12pt}\parbox[b]{0.18\textwidth-12pt-12pt}{\huxtpad{2pt + 1em}\centering 37\(\%\)\huxbpad{2pt}}} &
\multicolumn{1}{p{0.18\textwidth}!{\huxvb{0, 0, 0}{0}}}{\hspace{12pt}\parbox[b]{0.18\textwidth-12pt-12pt}{\huxtpad{2pt + 1em}\centering 593\huxbpad{2pt}}} &
\multicolumn{1}{p{0.18\textwidth}!{\huxvb{0, 0, 0}{0}}}{\hspace{12pt}\parbox[b]{0.18\textwidth-12pt-0pt}{\huxtpad{2pt + 1em}\centering 34\(\%\)\huxbpad{2pt}}} \tabularnewline[-0.5pt]


\hhline{}
\arrayrulecolor{black}

\multicolumn{1}{!{\huxvb{0, 0, 0}{0}}p{0.18\textwidth}!{\huxvb{0, 0, 0}{0}}}{\hspace{0pt}\parbox[b]{0.18\textwidth-0pt-12pt}{\huxtpad{2pt + 1em}\centering D\huxbpad{2pt}}} &
\multicolumn{1}{p{0.18\textwidth}!{\huxvb{0, 0, 0}{0}}}{\hspace{12pt}\parbox[b]{0.18\textwidth-12pt-12pt}{\huxtpad{2pt + 1em}\centering 417\huxbpad{2pt}}} &
\multicolumn{1}{p{0.18\textwidth}!{\huxvb{0, 0, 0}{0}}}{\hspace{12pt}\parbox[b]{0.18\textwidth-12pt-12pt}{\huxtpad{2pt + 1em}\centering 33\(\%\)\huxbpad{2pt}}} &
\multicolumn{1}{p{0.18\textwidth}!{\huxvb{0, 0, 0}{0}}}{\hspace{12pt}\parbox[b]{0.18\textwidth-12pt-12pt}{\huxtpad{2pt + 1em}\centering 375\huxbpad{2pt}}} &
\multicolumn{1}{p{0.18\textwidth}!{\huxvb{0, 0, 0}{0}}}{\hspace{12pt}\parbox[b]{0.18\textwidth-12pt-0pt}{\huxtpad{2pt + 1em}\centering 35\(\%\)\huxbpad{2pt}}} \tabularnewline[-0.5pt]


\hhline{}
\arrayrulecolor{black}

\multicolumn{1}{!{\huxvb{0, 0, 0}{0}}p{0.18\textwidth}!{\huxvb{0, 0, 0}{0}}}{\hspace{0pt}\parbox[b]{0.18\textwidth-0pt-12pt}{\huxtpad{2pt + 1em}\centering E\huxbpad{2pt}}} &
\multicolumn{1}{p{0.18\textwidth}!{\huxvb{0, 0, 0}{0}}}{\hspace{12pt}\parbox[b]{0.18\textwidth-12pt-12pt}{\huxtpad{2pt + 1em}\centering 191\huxbpad{2pt}}} &
\multicolumn{1}{p{0.18\textwidth}!{\huxvb{0, 0, 0}{0}}}{\hspace{12pt}\parbox[b]{0.18\textwidth-12pt-12pt}{\huxtpad{2pt + 1em}\centering 28\(\%\)\huxbpad{2pt}}} &
\multicolumn{1}{p{0.18\textwidth}!{\huxvb{0, 0, 0}{0}}}{\hspace{12pt}\parbox[b]{0.18\textwidth-12pt-12pt}{\huxtpad{2pt + 1em}\centering 393\huxbpad{2pt}}} &
\multicolumn{1}{p{0.18\textwidth}!{\huxvb{0, 0, 0}{0}}}{\hspace{12pt}\parbox[b]{0.18\textwidth-12pt-0pt}{\huxtpad{2pt + 1em}\centering 24\(\%\)\huxbpad{2pt}}} \tabularnewline[-0.5pt]


\hhline{}
\arrayrulecolor{black}

\multicolumn{1}{!{\huxvb{0, 0, 0}{0}}p{0.18\textwidth}!{\huxvb{0, 0, 0}{0}}}{\hspace{0pt}\parbox[b]{0.18\textwidth-0pt-12pt}{\huxtpad{2pt + 1em}\centering F\huxbpad{2pt}}} &
\multicolumn{1}{p{0.18\textwidth}!{\huxvb{0, 0, 0}{0}}}{\hspace{12pt}\parbox[b]{0.18\textwidth-12pt-12pt}{\huxtpad{2pt + 1em}\centering 272\huxbpad{2pt}}} &
\multicolumn{1}{p{0.18\textwidth}!{\huxvb{0, 0, 0}{0}}}{\hspace{12pt}\parbox[b]{0.18\textwidth-12pt-12pt}{\huxtpad{2pt + 1em}\centering 6\(\%\)\huxbpad{2pt}}} &
\multicolumn{1}{p{0.18\textwidth}!{\huxvb{0, 0, 0}{0}}}{\hspace{12pt}\parbox[b]{0.18\textwidth-12pt-12pt}{\huxtpad{2pt + 1em}\centering 341\huxbpad{2pt}}} &
\multicolumn{1}{p{0.18\textwidth}!{\huxvb{0, 0, 0}{0}}}{\hspace{12pt}\parbox[b]{0.18\textwidth-12pt-0pt}{\huxtpad{2pt + 1em}\centering 7\(\%\)\huxbpad{2pt}}} \tabularnewline[-0.5pt]


\hhline{>{\huxb{0, 0, 0}{0.4}}->{\huxb{0, 0, 0}{0.4}}->{\huxb{0, 0, 0}{0.4}}->{\huxb{0, 0, 0}{0.4}}->{\huxb{0, 0, 0}{0.4}}-}
\arrayrulecolor{black}
\end{tabularx} 

\end{threeparttable}\par\end{centerbox}

\end{table}
 

Remarkably, most departments had a \emph{higher} rate of admissions for
females than for males! Yet the overall rate of admission across the
university for females was lower than for males. How can this be? How
can both of these statements be true at the same time?

Here's what's going on. Firstly, notice that the departments are not
equal to one another in terms of their admission percentages: some
departments (e.g., A, B) tended to admit a high percentage of the
qualified applicants, whereas others (e.g., F) tended to reject most of
the candidates, even if they were high quality. So, among the six
departments shown above, notice that department A is the most generous,
followed by B, C, D, E and F in that order. Next, notice that males and
females tended to apply to different departments. If we rank the
departments in terms of the total number of male applicants, we get
\textbf{A}\textgreater{}\textbf{B}\textgreater D\textgreater C\textgreater F\textgreater E
(the ``easy'' departments are in bold). On the whole, males tended to
apply to the departments that had high admission rates. Now compare this
to how the female applicants distributed themselves. Ranking the
departments in terms of the total number of female applicants produces a
quite different ordering
C\textgreater E\textgreater D\textgreater F\textgreater{}\textbf{A}\textgreater{}\textbf{B}.
In other words, what these data seem to be suggesting is that the female
applicants tended to apply to ``harder'' departments. And in fact, if we
look at Figure~\ref{fig-fig1-1} we see that this trend is systematic,
and quite striking. This effect is known as \textbf{Simpson's paradox}.
It's not common, but it does happen in real life, and most people are
very surprised by it when they first encounter it, and many people
refuse to even believe that it's real. It is very real. And while there
are lots of very subtle statistical lessons buried in there, I want to
use it to make a much more important point: doing research is hard, and
there are lots of subtle, counter-intuitive traps lying in wait for the
unwary. That's reason \#2 why scientists love statistics, and why we
teach research methods. Because science is hard, and the truth is
sometimes cunningly hidden in the nooks and crannies of complicated
data.

\begin{figure}[h!]

\includegraphics[width=0.8\textwidth,height=\textheight]{01-Why-do-we-learn-statistics_files/figure-pdf/fig-fig1-1-1.pdf} \hfill{}

\caption{\label{fig-fig1-1}The Berkeley 1973 college admissions data.
This figure plots the admission rate for the 85 departments that had at
least one female applicant, as a function of the percentage of
applicants that were female. The plot is a redrawing of Figure 1 from
Bickel et al. (1975). Circles plot departments with more than 40
applicants; the area of the circle is proportional to the total number
of applicants. The crosses plot departments with fewer than 40
applicants}

\end{figure}

Before leaving this topic entirely, I want to point out something else
really critical that is often overlooked in a research methods class.
Statistics only solves \emph{part} of the problem. Remember that we
started all this with the concern that Berkeley's admissions processes
might be unfairly biased against female applicants. When we looked at
the ``aggregated'' data, it did seem like the university was
discriminating against women, but when we ``disaggregate'' and looked at
the individual behaviour of all the departments, it turned out that the
actual departments were, if anything, slightly biased in favour of
women. The gender bias in total admissions was caused by the fact that
women tended to self-select for harder departments. From a legal
perspective, that would probably put the university in the clear.
Postgraduate admissions are determined at the level of the individual
department, and there are good reasons to do that. At the level of
individual departments the decisions are more or less unbiased (the weak
bias in favour of females at that level is small, and not consistent
across departments). Since the university can't dictate which
departments people choose to apply to, and the decision making takes
place at the level of the department it can hardly be held accountable
for any biases that those choices produce.

That was the basis for my somewhat glib remarks earlier, but that's not
exactly the whole story, is it? After all, if we're interested in this
from a more sociological and psychological perspective, we might want to
ask \emph{why} there are such strong gender differences in applications.
Why do males tend to apply to engineering more often than females, and
why is this reversed for the English department? And why is it the case
that the departments that tend to have a female-application bias tend to
have lower overall admission rates than those departments that have a
male-application bias? Might this not still reflect a gender bias, even
though every single department is itself unbiased? It might. Suppose,
hypothetically, that males preferred to apply to ``hard sciences'' and
females prefer ``humanities''. And suppose further that the reason for
why the humanities departments have low admission rates is because the
government doesn't want to fund the humanities (Ph.D.~places, for
instance, are often tied to government funded research projects). Does
that constitute a gender bias? Or just an unenlightened view of the
value of the humanities? What if someone at a high level in the
government cut the humanities funds because they felt that the
humanities are ``useless chick stuff''. That seems pretty blatantly
gender biased. None of this falls within the purview of statistics, but
it matters to the research project. If you're interested in the overall
structural effects of subtle gender biases, then you probably want to
look at both the aggregated and disaggregated data. If you're interested
in the decision making process at Berkeley itself then you're probably
only interested in the disaggregated data.

In short there are a lot of critical questions that you can't answer
with statistics, but the answers to those questions will have a huge
impact on how you analyse and interpret data. And this is the reason why
you should always think of statistics as a tool to help you learn about
your data. No more and no less. It's a powerful tool to that end, but
there's no substitute for careful thought.

\hypertarget{statistics-in-psychology}{%
\section{Statistics in psychology}\label{statistics-in-psychology}}

I hope that the discussion above helped explain why science in general
is so focused on statistics. But I'm guessing that you have a lot more
questions about what role statistics plays in psychology, and
specifically why psychology classes always devote so many lectures to
stats. So here's my attempt to answer a few of them\ldots{}

\begin{quote}
\textbf{Why does psychology have so much statistics?}
\end{quote}

To be perfectly honest, there's a few different reasons, some of which
are better than others. The most important reason is that psychology is
a statistical science. What I mean by that is that the ``things'' that
we study are \emph{people}. Real, complicated, gloriously messy,
infuriatingly perverse people. The ``things'' of physics include objects
like electrons, and while there are all sorts of complexities that arise
in physics, electrons don't have minds of their own. They don't have
opinions, they don't differ from each other in weird and arbitrary ways,
they don't get bored in the middle of an experiment, and they don't get
angry at the experimenter and then deliberately try to sabotage the data
set (not that I've ever done that!). At a fundamental level psychology
is harder than physics.\footnote{Which might explain why physics is just
  a teensy bit further advanced as a science than we are.} Basically, we
teach statistics to you as psychologists because you need to be better
at stats than physicists. There's actually a saying used sometimes in
physics, to the effect that ``if your experiment needs statistics, you
should have done a better experiment''. They have the luxury of being
able to say that because their objects of study are pathetically simple
in comparison to the vast mess that confronts social scientists. And
it's not just psychology. Most social sciences are desperately reliant
on statistics. Not because we're bad experimenters, but because we've
picked a harder problem to solve. We teach you stats because you really,
really need it.

\begin{quote}
\textbf{Can't someone else do the statistics?}
\end{quote}

To some extent, but not completely. It's true that you don't need to
become a fully trained statistician just to do psychology, but you do
need to reach a certain level of statistical competence. In my view,
there's three reasons that every psychological researcher ought to be
able to do basic statistics:

\begin{itemize}
\tightlist
\item
  Firstly, there's the fundamental reason: statistics is deeply
  intertwined with research design. If you want to be good at designing
  psychological studies, you need to at the very least understand the
  basics of stats.
\item
  Secondly, if you want to be good at the psychological side of the
  research, then you need to be able to understand the psychological
  literature, right? But almost every paper in the psychological
  literature reports the results of statistical analyses. So if you
  really want to understand the psychology, you need to be able to
  understand what other people did with their data. And that means
  understanding a certain amount of statistics.
\item
  Thirdly, there's a big practical problem with being dependent on other
  people to do all your statistics: statistical analysis is
  \emph{expensive}. If you ever get bored and want to look up how much
  the Australian government charges for university fees, you'll notice
  something interesting: statistics is designated as a ``national
  priority'' category, and so the fees are much, much lower than for any
  other area of study. This is because there's a massive shortage of
  statisticians out there. So, from your perspective as a psychological
  researcher, the laws of supply and demand aren't exactly on your side
  here! As a result, in almost any real-life situation where you want to
  do psychological research, the cruel facts will be that you don't have
  enough money to afford a statistician. So the economics of the
  situation mean that you have to be pretty self-sufficient.
\end{itemize}

Note that a lot of these reasons generalise beyond researchers. If you
want to be a practicing psychologist and stay on top of the field, it
helps to be able to read the scientific literature, which relies pretty
heavily on statistics.

\begin{quote}
\textbf{I don't care about jobs, research or clinical work. Do I need
statistics?}
\end{quote}

Okay, now you're just messing with me. Still, I think it should matter
to you too. Statistics should matter to you in the same way that
statistics should matter to \emph{everyone}. We live in the 21st
century, and data are \emph{everywhere}. Frankly, given the world in
which we live these days, a basic knowledge of statistics is pretty damn
close to a survival tool! Which is the topic of the next section.

\hypertarget{statistics-in-everyday-life}{%
\section{Statistics in everyday
life}\label{statistics-in-everyday-life}}

\begin{quote}
\emph{We are drowning in information,}\\
\emph{but we are starved for knowledge}\\
-- Various authors, original probably John Naisbitt
\end{quote}

When I started writing up my lecture notes I took the 20 most recent
news articles posted to the ABC News website. Of those articles, eight
of them included a discussion of a statistical topic and six of those
made a mistake. The most common error was failing to report baseline
data (e.g., the article mentions that 5\% of people in situation X have
some characteristic Y, but doesn't say how common the characteristic is
for everyone else!). The point I'm trying to make here isn't that
journalists are bad at statistics (though they almost always are), it's
that a basic knowledge of statistics is very helpful for trying to
figure out when someone else is either making a mistake or even lying to
you. In fact, one of the biggest things that a knowledge of statistics
does to you is cause you to get angry at the newspaper or the internet
on a far more frequent basis. You can find a good example of this in
\textbf{?@sec-A-real-life-example} in
\textbf{?@sec-Descriptive-statistics}. In later versions of this book
I'll try to include more anecdotes along those lines.

\hypertarget{theres-more-to-research-methods-than-statistics}{%
\section{There's more to research methods than
statistics}\label{theres-more-to-research-methods-than-statistics}}

So far, most of what I've talked about is statistics, and so you'd be
forgiven for thinking that statistics is all I care about. To be fair,
you wouldn't be far wrong, but research methodology is a broader concept
than statistics. So most research methods courses will cover topics that
relate much more to the pragmatics of research design, and in particular
the issues that you encounter when trying to do research with humans.
However, about 99\% of student fears relate to the statistics part of
the course, so I've focused on the stats in this discussion, and
hopefully I've convinced you that statistics matter, and more
importantly, should not to be feared. That said, it's typical for
introductory research methods classes to be very stats heavy. This is
not (usually) because the lecturers are evil people. Quite the contrary,
in fact. Introductory classes focus a lot on the statistics because you
almost always find yourself needing statistics before you need other
research methods training. Why? Because almost all your assignments in
other classes will rely on statistical training, to a greater extent
than they rely on other methodological tools. It's not common for
undergraduate assignments to require you to design your own study from
the ground up (in which case you would need to know a lot about research
design), but it \emph{is} common for assignments to ask you to analyse
and interpret data that were collected in a study that someone else
designed (in which case you need statistics). In that sense, from the
perspective of enabling you to do well in all your other classes,
statistics is more urgent.

But note that ``urgent'' is different from ``important'' -- they both
matter. I really do want to stress that research design is just as
important as data analysis, and this book does spend some time on it.
However, while statistics has a kind of universality, and provides a set
of core tools that are useful for most types of psychological and social
research, the research methods side isn't quite so universal. There are
some general principles that everyone should think about, but a lot of
research design is idiosyncratic and is specific to the area of research
that you want to engage in.

\hypertarget{sec-A-brief-introduction-to-research-design}{%
\chapter{A brief introduction to research
design}\label{sec-A-brief-introduction-to-research-design}}

\placetextbox{0.26}{0.06}{\scriptsize{©2025 D. Foxcroft and D. Navarro,}}
\placetextbox{0.205}{0.05}{\scriptsize{CC BY-SA 4.0}}
\placetextbox{0.70}{0.06}{\scriptsize{\url{https://doi.org/10.11647/OBP.0333/02}}}

\begin{quote}
\emph{To consult the statistician after an experiment is finished is
often merely to ask him to conduct a post mortem examination. He can
perhaps say what the experiment died of.}\\
-- Sir Ronald Fisher\footnote{Presidential Address to the First Indian
  Statistical Congress, 1938. Source:
  \url{https://en.wikiquote.org/wiki/Ronald_Fisher}}
\end{quote}

In this chapter, we're going to start thinking about the basic ideas
that go into designing a study, collecting data, checking whether your
data collection works, and so on. It won't give you enough information
to allow you to design studies of your own, but it will give you a lot
of the basic tools that you need to assess the studies done by other
people. However, since the focus of this book is much more on data
analysis than on data collection, I'm only giving a very brief overview.
Note that this chapter is ``special'' in two ways. Firstly, it's much
more psychology specific than the later chapters. Secondly, it focuses
much more heavily on the scientific problem of research methodology, and
much less on the statistical problem of data analysis. Nevertheless, the
two problems are related to one another, so it's traditional for stats
textbooks to discuss the problem in a little detail. This chapter relies
heavily on Campbell \& Stanley (1963) and Stevens (1946) for the
discussion of scales of measurement.

\hypertarget{sec-Introduction-to-psychological-measurement}{%
\section{Introduction to psychological
measurement}\label{sec-Introduction-to-psychological-measurement}}

The first thing to understand is data collection can be thought of as a
kind of \textbf{measurement}. That is, what we're trying to do here is
measure something about human behaviour or the human mind. What do I
mean by ``measurement''?

\hypertarget{some-thoughts-about-psychological-measurement}{%
\subsection{Some thoughts about psychological
measurement}\label{some-thoughts-about-psychological-measurement}}

Measurement itself is a subtle concept, but basically it comes down to
finding some way of assigning numbers, or labels, or some other kind of
well-defined descriptions, to ``stuff''. So, any of the following would
count as a psychological measurement:

\begin{itemize}
\tightlist
\item
  My \textbf{age} is \emph{33} years.
\item
  I \emph{do not} \textbf{like anchovies.}
\item
  My \textbf{chromosomal gender} is \emph{male}.
\item
  My \textbf{self-identified gender} is \emph{female}.
\end{itemize}

In the short list above, the \textbf{bolded part} is ``the thing to be
measured'', and the \emph{italicised part} is ``the measurement
itself''. In fact, we can expand on this a little bit, by thinking about
the set of possible measurements that could have arisen in each case:

\begin{itemize}
\tightlist
\item
  My \textbf{age} (in years) could have been \emph{0, 1, 2, 3} \ldots,
  etc. The upper bound on what my age could possibly be is a bit fuzzy,
  but in practice you'd be safe in saying that the largest possible age
  is \emph{150}, since no human has ever lived that long.
\item
  When asked if I \textbf{like anchovies}, I might have said that
  \emph{I do}, or \emph{I do not}, or \emph{I have no opinion}, or
  \emph{I sometimes do.}
\item
  My \textbf{chromosomal gender} is almost certainly going to be
  \emph{male} (\(XY\)) or \emph{female} (\(XX\)), but there are a few
  other possibilities. I could also have \emph{Klinfelter's syndrome}
  (\(XXY\)), which is more similar to male than to female. And I imagine
  there are other possibilities too.
\item
  My \textbf{self-identified} gender is also very likely to be male or
  female, but it doesn't have to agree with my chromosomal gender. I may
  also choose to identify with \emph{neither}, or to explicitly call
  myself \emph{transgender}.
\end{itemize}

As you can see, for some things (like age) it seems fairly obvious what
the set of possible measurements should be, whereas for other things it
gets a bit tricky. But I want to point out that even in the case of
someone's age it's much more subtle than this. For instance, in the
example above I assumed that it was okay to measure age in years. But if
you're a developmental psychologist, that's way too crude, and so you
often measure age in \emph{years and months} (if a child is 2 years and
11 months this is usually written as ``2;11''). If you're interested in
newborns you might want to measure age in \emph{days since birth}, maybe
even \emph{hours since birth}. In other words, the way in which you
specify the allowable measurement values is important.

Looking at this a bit more closely, you might also realise that the
concept of ``age'' isn't actually all that precise. In general, when we
say ``age'' we implicitly mean ``the length of time since birth''. But
that's not always the right way to do it. Suppose you're interested in
how newborn babies control their eye movements. If you're interested in
kids that young, you might also start to worry that ``birth'' is not the
only meaningful point in time to care about. If Baby Alice is born 3
weeks premature and Baby Bianca is born 1 week late, would it really
make sense to say that they are the ``same age'' if we encountered them
``2 hours after birth''? In one sense, yes. By social convention we use
birth as our reference point for talking about age in everyday life,
since it defines the amount of time the person has been operating as an
independent entity in the world. But from a scientific perspective
that's not the only thing we care about. When we think about the biology
of human beings, it's often useful to think of ourselves as organisms
that have been growing and maturing since conception, and from that
perspective Alice and Bianca aren't the same age at all. So you might
want to define the concept of ``age'' in two different ways: the length
of time since conception and the length of time since birth. When
dealing with adults it won't make much difference, but when dealing with
newborns it might.

Moving beyond these issues, there's the question of methodology. What
specific ``measurement method'' are you going to use to find out
someone's age? As before, there are lots of different possibilities:

\begin{itemize}
\tightlist
\item
  You could just ask people ``how old are you?'' The method of
  self-report is fast, cheap and easy. But it only works with people old
  enough to understand the question, and some people lie about their
  age.
\item
  You could ask an authority (e.g., a parent) ``how old is your child?''
  This method is fast, and when dealing with kids it's not all that hard
  since the parent is almost always around. It doesn't work as well if
  you want to know ``age since conception'', since a lot of parents
  can't say for sure when conception took place. For that, you might
  need a different authority (e.g., an obstetrician).
\item
  You could look up official records, for example birth or death
  certificates. This is a time consuming and frustrating endeavour, but
  it has its uses (e.g., if the person is now dead).
\end{itemize}

\hypertarget{operationalisation-defining-your-measurement}{%
\subsection{Operationalisation: defining your
measurement}\label{operationalisation-defining-your-measurement}}

All of the ideas discussed in the previous section relate to the concept
of \textbf{operationalisation}. To be a bit more precise about the idea,
operationalisation is the process by which we take a meaningful but
somewhat vague concept and turn it into a precise measurement. The
process of operationalisation can involve several different things:

\begin{itemize}
\item
  Being precise about what you are trying to measure. For instance, does
  ``age'' mean ``time since birth'' or ``time since conception'' in the
  context of your research?
\item
  Determining what method you will use to measure it. Will you use
  self-report to measure age, ask a parent, or look up an official
  record? If you're using self-report, how will you phrase the question?
\item
  Defining the set of allowable values that the measurement can take.
  Note that these values don't always have to be numerical, though they
  often are. When measuring age the values are numerical, but we still
  need to think carefully about what numbers are allowed. Do we want age
  in years, years and months, days, or hours? For other types of
  measurements (e.g., gender) the values aren't numerical. But, just as
  before, we need to think about what values are allowed. If we're
  asking people to self-report their gender, what options do we allow
  them to choose between? Is it enough to allow only ``male'' or
  ``female''? Do you need an ``other'' option? Or should we not give
  people specific options and instead let them answer in their own
  words? And if you open up the set of possible values to include all
  verbal response, how will you interpret their answers?
\end{itemize}

Operationalisation is a tricky business, and there's no ``one, true
way'' to do it. The way in which you choose to operationalise the
informal concept of ``age'' or ``gender'' into a formal measurement
depends on what you need to use the measurement for. Often you'll find
that the community of scientists who work in your area have some fairly
well-established ideas for how to go about it. In other words,
operationalisation needs to be thought through on a case-by-case basis.
Nevertheless, while there are a lot of issues that are specific to each
individual research project, there are some aspects to it that are
pretty general.

Before moving on I want to take a moment to clear up our terminology,
and in the process introduce one more term. Here are four different
things that are closely related to each other:

\begin{itemize}
\tightlist
\item
  \textbf{A theoretical construct.} This is the thing that you're trying
  to take a measurement of, like ``age'', ``gender'' or an ``opinion''.
  A theoretical construct can't be directly observed, and often they're
  actually a bit vague.
\item
  \textbf{A measure.} The measure refers to the method or the tool that
  you use to make your observations. A question in a survey, a
  behavioural observation or a brain scan could all count as a measure.
\item
  \textbf{An operationalisation.} The term ``operationalisation'' refers
  to the logical connection between the measure and the theoretical
  construct, or to the process by which we try to derive a measure from
  a theoretical construct.
\item
  \textbf{A variable.} Finally, a new term. A variable is what we end up
  with when we apply our measure to something in the world. That is,
  variables are the actual ``data'' that we end up with in our data
  sets.
\end{itemize}

In practice, even scientists tend to blur the distinction between these
things, but it's very helpful to try to understand the differences.

\hypertarget{sec-Scales-of-measurement}{%
\section{Scales of measurement}\label{sec-Scales-of-measurement}}

As the previous section indicates, the outcome of a psychological
measurement is called a variable. But not all variables are of the same
qualitative type and so it's useful to understand what types there are.
A very useful concept for distinguishing between different types of
variables is what's known as \textbf{scales of measurement}.

\hypertarget{nominal-scale}{%
\subsection{Nominal scale}\label{nominal-scale}}

A \textbf{nominal scale} variable (also referred to as a
\textbf{categorical} variable) is one in which there is no particular
relationship between the different possibilities. For these kinds of
variables it doesn't make any sense to say that one of them is
``bigger'' or ``better'' than any other one, and it absolutely doesn't
make any sense to average them. The classic example for this is ``eye
colour''. Eyes can be blue, green or brown, amongst other possibilities,
but none of them is any ``bigger'' than any other one. As a result, it
would feel really weird to talk about an ``average eye colour''.
Similarly, gender is nominal too: male isn't better or worse than
female. Neither does it make sense to try to talk about an ``average
gender''. In short, nominal scale variables are those for which the only
thing you can say about the different possibilities is that they are
different. That's it.

Let's take a slightly closer look at this. Suppose I was doing research
on how people commute to and from work. One variable I would have to
measure would be what kind of transportation people use to get to work.
This ``transport type'' variable could have quite a few possible values,
including: ``train'', ``bus'', ``car'', ``bicycle''. For now, let's
suppose that these four are the only possibilities. Then imagine that I
ask 100 people how they got to work today, with this result
(Table~\ref{tbl-tab2-1}).

\hypertarget{tbl-tab2-1}{}
 
  \providecommand{\huxb}[2]{\arrayrulecolor[RGB]{#1}\global\arrayrulewidth=#2pt}
  \providecommand{\huxvb}[2]{\color[RGB]{#1}\vrule width #2pt}
  \providecommand{\huxtpad}[1]{\rule{0pt}{#1}}
  \providecommand{\huxbpad}[1]{\rule[-#1]{0pt}{#1}}

\begin{table}[ht]
\caption{\label{tbl-tab2-1}How did 100 people get to work today }\tabularnewline

\begin{centerbox}
\begin{threeparttable}
\setlength{\tabcolsep}{0pt}
\begin{tabularx}{0.9\textwidth}{p{0.45\textwidth} p{0.45\textwidth}}


\hhline{>{\huxb{0, 0, 0}{0.4}}->{\huxb{0, 0, 0}{0.4}}-}
\arrayrulecolor{black}

\multicolumn{1}{!{\huxvb{0, 0, 0}{0}}p{0.45\textwidth}!{\huxvb{0, 0, 0}{0}}}{\hspace{0pt}\parbox[b]{0.45\textwidth-0pt-12pt}{\huxtpad{2pt + 1em}\centering \textbf{Transportation}\huxbpad{2pt}}} &
\multicolumn{1}{p{0.45\textwidth}!{\huxvb{0, 0, 0}{0}}}{\hspace{12pt}\parbox[b]{0.45\textwidth-12pt-0pt}{\huxtpad{2pt + 1em}\centering \textbf{Number of people}\huxbpad{2pt}}} \tabularnewline[-0.5pt]


\hhline{>{\huxb{0, 0, 0}{0.4}}->{\huxb{0, 0, 0}{0.4}}-}
\arrayrulecolor{black}

\multicolumn{1}{!{\huxvb{0, 0, 0}{0}}p{0.45\textwidth}!{\huxvb{0, 0, 0}{0}}}{\hspace{0pt}\parbox[b]{0.45\textwidth-0pt-12pt}{\huxtpad{2pt + 1em}\centering (1) Train\huxbpad{2pt}}} &
\multicolumn{1}{p{0.45\textwidth}!{\huxvb{0, 0, 0}{0}}}{\hspace{12pt}\parbox[b]{0.45\textwidth-12pt-0pt}{\huxtpad{2pt + 1em}\centering 12\huxbpad{2pt}}} \tabularnewline[-0.5pt]


\hhline{}
\arrayrulecolor{black}

\multicolumn{1}{!{\huxvb{0, 0, 0}{0}}p{0.45\textwidth}!{\huxvb{0, 0, 0}{0}}}{\hspace{0pt}\parbox[b]{0.45\textwidth-0pt-12pt}{\huxtpad{2pt + 1em}\centering (2) Bus\huxbpad{2pt}}} &
\multicolumn{1}{p{0.45\textwidth}!{\huxvb{0, 0, 0}{0}}}{\hspace{12pt}\parbox[b]{0.45\textwidth-12pt-0pt}{\huxtpad{2pt + 1em}\centering 30\huxbpad{2pt}}} \tabularnewline[-0.5pt]


\hhline{}
\arrayrulecolor{black}

\multicolumn{1}{!{\huxvb{0, 0, 0}{0}}p{0.45\textwidth}!{\huxvb{0, 0, 0}{0}}}{\hspace{0pt}\parbox[b]{0.45\textwidth-0pt-12pt}{\huxtpad{2pt + 1em}\centering (3) Car\huxbpad{2pt}}} &
\multicolumn{1}{p{0.45\textwidth}!{\huxvb{0, 0, 0}{0}}}{\hspace{12pt}\parbox[b]{0.45\textwidth-12pt-0pt}{\huxtpad{2pt + 1em}\centering 48\huxbpad{2pt}}} \tabularnewline[-0.5pt]


\hhline{}
\arrayrulecolor{black}

\multicolumn{1}{!{\huxvb{0, 0, 0}{0}}p{0.45\textwidth}!{\huxvb{0, 0, 0}{0}}}{\hspace{0pt}\parbox[b]{0.45\textwidth-0pt-12pt}{\huxtpad{2pt + 1em}\centering (4) Bicycle\huxbpad{2pt}}} &
\multicolumn{1}{p{0.45\textwidth}!{\huxvb{0, 0, 0}{0}}}{\hspace{12pt}\parbox[b]{0.45\textwidth-12pt-0pt}{\huxtpad{2pt + 1em}\centering 10\huxbpad{2pt}}} \tabularnewline[-0.5pt]


\hhline{>{\huxb{0, 0, 0}{0.4}}->{\huxb{0, 0, 0}{0.4}}-}
\arrayrulecolor{black}
\end{tabularx} 

\end{threeparttable}\par\end{centerbox}

\end{table}
 

So, what's the average transportation type? Obviously, the answer here
is that there isn't one. It's a silly question to ask. You can say that
travel by car is the most popular method, and travel by train is the
least popular method, but that's about all. Similarly, notice that the
order in which I list the options isn't very interesting. I could have
chosen to display the data like in Table~\ref{tbl-tab2-2}.

\hypertarget{tbl-tab2-2}{}
 
  \providecommand{\huxb}[2]{\arrayrulecolor[RGB]{#1}\global\arrayrulewidth=#2pt}
  \providecommand{\huxvb}[2]{\color[RGB]{#1}\vrule width #2pt}
  \providecommand{\huxtpad}[1]{\rule{0pt}{#1}}
  \providecommand{\huxbpad}[1]{\rule[-#1]{0pt}{#1}}

\begin{table}[h!]
\caption{\label{tbl-tab2-2}How did 100 people get to work today, a different view }\tabularnewline

\begin{centerbox}
\begin{threeparttable}
\setlength{\tabcolsep}{0pt}
\begin{tabularx}{0.9\textwidth}{p{0.45\textwidth} p{0.45\textwidth}}


\hhline{>{\huxb{0, 0, 0}{0.4}}->{\huxb{0, 0, 0}{0.4}}-}
\arrayrulecolor{black}

\multicolumn{1}{!{\huxvb{0, 0, 0}{0}}p{0.45\textwidth}!{\huxvb{0, 0, 0}{0}}}{\hspace{0pt}\parbox[b]{0.45\textwidth-0pt-12pt}{\huxtpad{2pt + 1em}\centering \textbf{Transportation}\huxbpad{2pt}}} &
\multicolumn{1}{p{0.45\textwidth}!{\huxvb{0, 0, 0}{0}}}{\hspace{12pt}\parbox[b]{0.45\textwidth-12pt-0pt}{\huxtpad{2pt + 1em}\centering \textbf{Number of people}\huxbpad{2pt}}} \tabularnewline[-0.5pt]


\hhline{>{\huxb{0, 0, 0}{0.4}}->{\huxb{0, 0, 0}{0.4}}-}
\arrayrulecolor{black}

\multicolumn{1}{!{\huxvb{0, 0, 0}{0}}p{0.45\textwidth}!{\huxvb{0, 0, 0}{0}}}{\hspace{0pt}\parbox[b]{0.45\textwidth-0pt-12pt}{\huxtpad{2pt + 1em}\centering (3) Car\huxbpad{2pt}}} &
\multicolumn{1}{p{0.45\textwidth}!{\huxvb{0, 0, 0}{0}}}{\hspace{12pt}\parbox[b]{0.45\textwidth-12pt-0pt}{\huxtpad{2pt + 1em}\centering 48\huxbpad{2pt}}} \tabularnewline[-0.5pt]


\hhline{}
\arrayrulecolor{black}

\multicolumn{1}{!{\huxvb{0, 0, 0}{0}}p{0.45\textwidth}!{\huxvb{0, 0, 0}{0}}}{\hspace{0pt}\parbox[b]{0.45\textwidth-0pt-12pt}{\huxtpad{2pt + 1em}\centering (1) Train\huxbpad{2pt}}} &
\multicolumn{1}{p{0.45\textwidth}!{\huxvb{0, 0, 0}{0}}}{\hspace{12pt}\parbox[b]{0.45\textwidth-12pt-0pt}{\huxtpad{2pt + 1em}\centering 12\huxbpad{2pt}}} \tabularnewline[-0.5pt]


\hhline{}
\arrayrulecolor{black}

\multicolumn{1}{!{\huxvb{0, 0, 0}{0}}p{0.45\textwidth}!{\huxvb{0, 0, 0}{0}}}{\hspace{0pt}\parbox[b]{0.45\textwidth-0pt-12pt}{\huxtpad{2pt + 1em}\centering (4) Bicycle\huxbpad{2pt}}} &
\multicolumn{1}{p{0.45\textwidth}!{\huxvb{0, 0, 0}{0}}}{\hspace{12pt}\parbox[b]{0.45\textwidth-12pt-0pt}{\huxtpad{2pt + 1em}\centering 10\huxbpad{2pt}}} \tabularnewline[-0.5pt]


\hhline{}
\arrayrulecolor{black}

\multicolumn{1}{!{\huxvb{0, 0, 0}{0}}p{0.45\textwidth}!{\huxvb{0, 0, 0}{0}}}{\hspace{0pt}\parbox[b]{0.45\textwidth-0pt-12pt}{\huxtpad{2pt + 1em}\centering (2) Bus\huxbpad{2pt}}} &
\multicolumn{1}{p{0.45\textwidth}!{\huxvb{0, 0, 0}{0}}}{\hspace{12pt}\parbox[b]{0.45\textwidth-12pt-0pt}{\huxtpad{2pt + 1em}\centering 30\huxbpad{2pt}}} \tabularnewline[-0.5pt]


\hhline{>{\huxb{0, 0, 0}{0.4}}->{\huxb{0, 0, 0}{0.4}}-}
\arrayrulecolor{black}
\end{tabularx} 

\end{threeparttable}\par\end{centerbox}

\end{table}
 

\ldots and nothing really changes.

\hypertarget{ordinal-scale}{%
\subsection{Ordinal scale}\label{ordinal-scale}}

\textbf{Ordinal scale} variables have a bit more structure than nominal
scale variables, but not by a lot. An ordinal scale variable is one in
which there is a natural, meaningful way to order the different
possibilities, but you can't do anything else. The usual example given
of an ordinal variable is ``finishing position in a race''. You
\emph{can} say that the person who finished first was faster than the
person who finished second, but you \emph{do not} know how much faster.
As a consequence we know that 1st \(>\) 2nd, and we know that 2nd \(>\)
3rd, but the difference between 1st and 2nd might be much larger than
the difference between 2nd and 3rd.

Here's a more psychologically interesting example. Suppose I'm
interested in people's attitudes to climate change. I then go and ask
some people to pick the statement (from four listed statements) that
most closely matches their beliefs:

\begin{enumerate}
\def\labelenumi{\arabic{enumi}.}
\tightlist
\item
  Temperatures are rising because of human activity
\item
  Temperatures are rising but we don't know why
\item
  Temperatures are rising but not because of humans
\item
  Temperatures are not rising
\end{enumerate}

Notice that these four statements actually do have a natural ordering,
in terms of ``the extent to which they agree with the current science''.
Statement 1 is a close match, statement 2 is a reasonable match,
statement 3 isn't a very good match, and statement 4 is in strong
opposition to current science. So, in terms of the thing I'm interested
in (the extent to which people endorse the science), I can order the
items as 1 \(>\) 2 \(>\) 3 \(>\) 4. Since this ordering exists, it would
be very weird to list the options like this\ldots{}

\begin{enumerate}
\def\labelenumi{\arabic{enumi}.}
\tightlist
\item
  Temperatures are rising but not because of humans
\item
  Temperatures are rising because of human activity
\item
  Temperatures are not rising
\item
  Temperatures are rising but we don't know why
\end{enumerate}

\ldots because it seems to violate the natural ``structure'' to the
question.

So, let's suppose I asked 100 people these questions, and got the
answers shown in Table~\ref{tbl-tab2-3}.

\hypertarget{tbl-tab2-3}{}
 
  \providecommand{\huxb}[2]{\arrayrulecolor[RGB]{#1}\global\arrayrulewidth=#2pt}
  \providecommand{\huxvb}[2]{\color[RGB]{#1}\vrule width #2pt}
  \providecommand{\huxtpad}[1]{\rule{0pt}{#1}}
  \providecommand{\huxbpad}[1]{\rule[-#1]{0pt}{#1}}

\begin{table}[ht]
\caption{\label{tbl-tab2-3}Attitudes to climate change }\tabularnewline

\begin{centerbox}
\begin{threeparttable}
\setlength{\tabcolsep}{0pt}
\begin{tabularx}{0.9\textwidth}{p{0.45\textwidth} p{0.45\textwidth}}


\hhline{>{\huxb{0, 0, 0}{0.4}}->{\huxb{0, 0, 0}{0.4}}-}
\arrayrulecolor{black}

\multicolumn{1}{!{\huxvb{0, 0, 0}{0}}p{0.45\textwidth}!{\huxvb{0, 0, 0}{0}}}{\hspace{0pt}\parbox[b]{0.45\textwidth-0pt-12pt}{\huxtpad{2pt + 1em}\centering \textbf{Response}\huxbpad{2pt}}} &
\multicolumn{1}{p{0.45\textwidth}!{\huxvb{0, 0, 0}{0}}}{\hspace{12pt}\parbox[b]{0.45\textwidth-12pt-0pt}{\huxtpad{2pt + 1em}\centering \textbf{Number}\huxbpad{2pt}}} \tabularnewline[-0.5pt]


\hhline{>{\huxb{0, 0, 0}{0.4}}->{\huxb{0, 0, 0}{0.4}}-}
\arrayrulecolor{black}

\multicolumn{1}{!{\huxvb{0, 0, 0}{0}}p{0.45\textwidth}!{\huxvb{0, 0, 0}{0}}}{\hspace{0pt}\parbox[b]{0.45\textwidth-0pt-12pt}{\huxtpad{2pt + 1em}\centering (1) Temperatures are rising because of human activity\huxbpad{2pt}}} &
\multicolumn{1}{p{0.45\textwidth}!{\huxvb{0, 0, 0}{0}}}{\hspace{12pt}\parbox[b]{0.45\textwidth-12pt-0pt}{\huxtpad{2pt + 1em}\centering 51\huxbpad{2pt}}} \tabularnewline[-0.5pt]


\hhline{}
\arrayrulecolor{black}

\multicolumn{1}{!{\huxvb{0, 0, 0}{0}}p{0.45\textwidth}!{\huxvb{0, 0, 0}{0}}}{\hspace{0pt}\parbox[b]{0.45\textwidth-0pt-12pt}{\huxtpad{2pt + 1em}\centering (2) Temperatures are rising but we do not know why\huxbpad{2pt}}} &
\multicolumn{1}{p{0.45\textwidth}!{\huxvb{0, 0, 0}{0}}}{\hspace{12pt}\parbox[b]{0.45\textwidth-12pt-0pt}{\huxtpad{2pt + 1em}\centering 20\huxbpad{2pt}}} \tabularnewline[-0.5pt]


\hhline{}
\arrayrulecolor{black}

\multicolumn{1}{!{\huxvb{0, 0, 0}{0}}p{0.45\textwidth}!{\huxvb{0, 0, 0}{0}}}{\hspace{0pt}\parbox[b]{0.45\textwidth-0pt-12pt}{\huxtpad{2pt + 1em}\centering (3) Temperatures are rising but not because of humans\huxbpad{2pt}}} &
\multicolumn{1}{p{0.45\textwidth}!{\huxvb{0, 0, 0}{0}}}{\hspace{12pt}\parbox[b]{0.45\textwidth-12pt-0pt}{\huxtpad{2pt + 1em}\centering 10\huxbpad{2pt}}} \tabularnewline[-0.5pt]


\hhline{}
\arrayrulecolor{black}

\multicolumn{1}{!{\huxvb{0, 0, 0}{0}}p{0.45\textwidth}!{\huxvb{0, 0, 0}{0}}}{\hspace{0pt}\parbox[b]{0.45\textwidth-0pt-12pt}{\huxtpad{2pt + 1em}\centering (4) Temperatures are not rising\huxbpad{2pt}}} &
\multicolumn{1}{p{0.45\textwidth}!{\huxvb{0, 0, 0}{0}}}{\hspace{12pt}\parbox[b]{0.45\textwidth-12pt-0pt}{\huxtpad{2pt + 1em}\centering 19\huxbpad{2pt}}} \tabularnewline[-0.5pt]


\hhline{>{\huxb{0, 0, 0}{0.4}}->{\huxb{0, 0, 0}{0.4}}-}
\arrayrulecolor{black}
\end{tabularx} 

\end{threeparttable}\par\end{centerbox}

\end{table}
 

When analysing these data it seems quite reasonable to try to group (1),
(2) and (3) together, and say that 81 out of 100 people were willing to
at \emph{least partially} endorse the science. And it's also quite
reasonable to group (2), (3) and (4) together and say that 49 out of 100
people registered \emph{at least some disagreement} with the dominant
scientific view. However, it would be entirely bizarre to try to group
(1), (2) and (4) together and say that 90 out of 100 people said\ldots{}
what? There's nothing sensible that allows you to group those responses
together at all.

That said, notice that while we \emph{can} use the natural ordering of
these items to construct sensible groupings, what we can't do is average
them. For instance, in my simple example here, the ``average'' response
to the question is 1.97. If you can tell me what that means I'd love to
know, because it seems like gibberish to me!

\hypertarget{interval-scale}{%
\subsection{Interval scale}\label{interval-scale}}

In contrast to nominal and ordinal scale variables, \textbf{interval
scale} and ratio scale variables are variables for which the numerical
value is genuinely meaningful. In the case of interval scale variables
the \emph{differences} between the numbers are interpretable, but the
variable doesn't have a ``natural'' zero value. A good example of an
interval scale variable is measuring temperature in degrees celsius. For
instance, if it was 15\(^{\circ}\) yesterday and 18\(^{\circ}\) today,
then the 3\(^{\circ}\) difference between the two is genuinely
meaningful. Moreover, that 3\(^{\circ}\) difference is \emph{exactly the
same} as the 3\(^{\circ}\) difference between 7\(^{\circ}\) and
10\(^{\circ}\). In short, addition and subtraction are meaningful for
interval scale variables.\footnote{Actually, I've been informed by
  readers with greater physics knowledge than I that temperature isn't
  strictly an interval scale, in the sense that the amount of energy
  required to heat something up by 3° depends on its current
  temperature. So in the sense that physicists care about, temperature
  isn't actually an interval scale. But it still makes a cute example so
  I'm going to ignore this little inconvenient truth.}

However, notice that the 0\(^{\circ}\) does not mean ``no temperature at
all''. It actually means ``the temperature at which water freezes'',
which is pretty arbitrary. As a consequence it becomes pointless to try
to multiply and divide temperatures. It is wrong to say that
20\(^{\circ}\) is twice as hot as 10\(^{\circ}\), just as it is weird
and meaningless to try to claim that 20\(^{\circ}\) is negative two
times as hot as -10\(^{\circ}\).

Again, lets look at a more psychological example. Suppose I'm interested
in looking at how the attitudes of first-year university students have
changed over time. Obviously, I'm going to want to record the year in
which each student started. This is an interval scale variable. A
student who started in 2003 did arrive 5 years before a student who
started in 2008. However, it would be completely daft for me to divide
2008 by 2003 and say that the second student started ``1.0024 times
later'' than the first one. That doesn't make any sense at all.

\hypertarget{ratio-scale}{%
\subsection{Ratio scale}\label{ratio-scale}}

The fourth and final type of variable to consider is a \textbf{ratio
scale} variable, in which zero really means zero, and it's okay to
multiply and divide. A good psychological example of a ratio scale
variable is response time (RT). In a lot of tasks it's very common to
record the amount of time somebody takes to solve a problem or answer a
question, because it's an indicator of how difficult the task is.
Suppose that Alan takes 2.3 seconds to respond to a question, whereas
Ben takes 3.1 seconds. As with an interval scale variable, addition and
subtraction are both meaningful here. Ben really did take 3.1 - 2.3 =
0.8 seconds longer than Alan did. However, notice that multiplication
and division also make sense here too: Ben took 3.1/2.3 = 1.35 times as
long as Alan did to answer the question. And the reason why you can do
this is that, for a ratio scale variable such as RT, ``zero seconds''
really does mean ``no time at all''.

\hypertarget{continuous-versus-discrete-variables}{%
\subsection{Continuous versus discrete
variables}\label{continuous-versus-discrete-variables}}

There's a second kind of distinction that you need to be aware of,
regarding what types of variables you can run into. This is the
distinction between continuous variables and discrete variables
(Table~\ref{tbl-tab2-4}). The difference between these is as follows:

\begin{itemize}
\tightlist
\item
  A \textbf{continuous variable} is one in which, for any two values
  that you can think of, it's always logically possible to have another
  value in between.
\item
  A \textbf{discrete variable} is, in effect, a variable that isn't
  continuous. For a discrete variable it's sometimes the case that
  there's nothing in the middle.
\end{itemize}

These definitions probably seem a bit abstract, but they're pretty
simple once you see some examples. For instance, response time is
continuous. If Alan takes 3.1 seconds and Ben takes 2.3 seconds to
respond to a question, then Cameron's response time will lie in between
if he took 3.0 seconds. And of course it would also be possible for
David to take 3.031 seconds to respond, meaning that his RT would lie in
between Cameron's and Alan's. And while in practice it might be
impossible to measure RT that precisely, it's certainly possible in
principle. Because we can always find a new value for RT in between any
two other ones we regard RT as a continuous measure.

\hypertarget{tbl-tab2-4}{}
 
  \providecommand{\huxb}[2]{\arrayrulecolor[RGB]{#1}\global\arrayrulewidth=#2pt}
  \providecommand{\huxvb}[2]{\color[RGB]{#1}\vrule width #2pt}
  \providecommand{\huxtpad}[1]{\rule{0pt}{#1}}
  \providecommand{\huxbpad}[1]{\rule[-#1]{0pt}{#1}}

\begin{table}[h!]
\caption{\label{tbl-tab2-4}The relationship between the scales of measurement and the
discrete/continuity distinction. Cells with a tick mark correspond to
things that are possible }\tabularnewline

\begin{centerbox}
\begin{threeparttable}
\setlength{\tabcolsep}{0pt}
\begin{tabularx}{0.9\textwidth}{p{0.3\textwidth} p{0.3\textwidth} p{0.3\textwidth}}


\hhline{>{\huxb{0, 0, 0}{0.4}}->{\huxb{0, 0, 0}{0.4}}->{\huxb{0, 0, 0}{0.4}}-}
\arrayrulecolor{black}

\multicolumn{1}{!{\huxvb{0, 0, 0}{0}}p{0.3\textwidth}!{\huxvb{0, 0, 0}{0}}}{\hspace{0pt}\parbox[b]{0.3\textwidth-0pt-12pt}{\huxtpad{2pt + 1em}\centering \textbf{}\huxbpad{2pt}}} &
\multicolumn{1}{p{0.3\textwidth}!{\huxvb{0, 0, 0}{0}}}{\hspace{12pt}\parbox[b]{0.3\textwidth-12pt-12pt}{\huxtpad{2pt + 1em}\centering \textbf{continuous}\huxbpad{2pt}}} &
\multicolumn{1}{p{0.3\textwidth}!{\huxvb{0, 0, 0}{0}}}{\hspace{12pt}\parbox[b]{0.3\textwidth-12pt-0pt}{\huxtpad{2pt + 1em}\centering \textbf{discrete}\huxbpad{2pt}}} \tabularnewline[-0.5pt]


\hhline{>{\huxb{0, 0, 0}{0.4}}->{\huxb{0, 0, 0}{0.4}}->{\huxb{0, 0, 0}{0.4}}-}
\arrayrulecolor{black}

\multicolumn{1}{!{\huxvb{0, 0, 0}{0}}p{0.3\textwidth}!{\huxvb{0, 0, 0}{0}}}{\hspace{0pt}\parbox[b]{0.3\textwidth-0pt-12pt}{\huxtpad{2pt + 1em}\centering nominal\huxbpad{2pt}}} &
\multicolumn{1}{p{0.3\textwidth}!{\huxvb{0, 0, 0}{0}}}{\hspace{12pt}\parbox[b]{0.3\textwidth-12pt-12pt}{\huxtpad{2pt + 1em}\centering \huxbpad{2pt}}} &
\multicolumn{1}{p{0.3\textwidth}!{\huxvb{0, 0, 0}{0}}}{\hspace{12pt}\parbox[b]{0.3\textwidth-12pt-0pt}{\huxtpad{2pt + 1em}\centering \( \checkmark \)\huxbpad{2pt}}} \tabularnewline[-0.5pt]


\hhline{}
\arrayrulecolor{black}

\multicolumn{1}{!{\huxvb{0, 0, 0}{0}}p{0.3\textwidth}!{\huxvb{0, 0, 0}{0}}}{\hspace{0pt}\parbox[b]{0.3\textwidth-0pt-12pt}{\huxtpad{2pt + 1em}\centering ordinal\huxbpad{2pt}}} &
\multicolumn{1}{p{0.3\textwidth}!{\huxvb{0, 0, 0}{0}}}{\hspace{12pt}\parbox[b]{0.3\textwidth-12pt-12pt}{\huxtpad{2pt + 1em}\centering \huxbpad{2pt}}} &
\multicolumn{1}{p{0.3\textwidth}!{\huxvb{0, 0, 0}{0}}}{\hspace{12pt}\parbox[b]{0.3\textwidth-12pt-0pt}{\huxtpad{2pt + 1em}\centering \( \checkmark \)\huxbpad{2pt}}} \tabularnewline[-0.5pt]


\hhline{}
\arrayrulecolor{black}

\multicolumn{1}{!{\huxvb{0, 0, 0}{0}}p{0.3\textwidth}!{\huxvb{0, 0, 0}{0}}}{\hspace{0pt}\parbox[b]{0.3\textwidth-0pt-12pt}{\huxtpad{2pt + 1em}\centering interval\huxbpad{2pt}}} &
\multicolumn{1}{p{0.3\textwidth}!{\huxvb{0, 0, 0}{0}}}{\hspace{12pt}\parbox[b]{0.3\textwidth-12pt-12pt}{\huxtpad{2pt + 1em}\centering \( \checkmark \)\huxbpad{2pt}}} &
\multicolumn{1}{p{0.3\textwidth}!{\huxvb{0, 0, 0}{0}}}{\hspace{12pt}\parbox[b]{0.3\textwidth-12pt-0pt}{\huxtpad{2pt + 1em}\centering \( \checkmark \)\huxbpad{2pt}}} \tabularnewline[-0.5pt]


\hhline{}
\arrayrulecolor{black}

\multicolumn{1}{!{\huxvb{0, 0, 0}{0}}p{0.3\textwidth}!{\huxvb{0, 0, 0}{0}}}{\hspace{0pt}\parbox[b]{0.3\textwidth-0pt-12pt}{\huxtpad{2pt + 1em}\centering ratio\huxbpad{2pt}}} &
\multicolumn{1}{p{0.3\textwidth}!{\huxvb{0, 0, 0}{0}}}{\hspace{12pt}\parbox[b]{0.3\textwidth-12pt-12pt}{\huxtpad{2pt + 1em}\centering \( \checkmark \)\huxbpad{2pt}}} &
\multicolumn{1}{p{0.3\textwidth}!{\huxvb{0, 0, 0}{0}}}{\hspace{12pt}\parbox[b]{0.3\textwidth-12pt-0pt}{\huxtpad{2pt + 1em}\centering \( \checkmark \)\huxbpad{2pt}}} \tabularnewline[-0.5pt]


\hhline{>{\huxb{0, 0, 0}{0.4}}->{\huxb{0, 0, 0}{0.4}}->{\huxb{0, 0, 0}{0.4}}-}
\arrayrulecolor{black}
\end{tabularx} 

\end{threeparttable}\par\end{centerbox}

\end{table}
 

Discrete variables occur when this rule is violated. For example,
nominal scale variables are always discrete. There isn't a type of
transportation that falls ``in between'' trains and bicycles, not in the
strict mathematical way that 2.3 falls in between 2 and 3. So
transportation type is discrete. Similarly, ordinal scale variables are
always discrete. Although ``2nd place'' does fall between ``1st place''
and ``3rd place'', there's nothing that can logically fall in between
``1st place'' and ``2nd place''. Interval scale and ratio scale
variables can go either way. As we saw above, response time (a ratio
scale variable) is continuous. Temperature in degrees celsius (an
interval scale variable) is also continuous. However, the year you went
to school (an interval scale variable) is discrete. There's no year in
between 2002 and 2003. The number of questions you get right on a
true-or-false test (a ratio scale variable) is also discrete. Since a
true-or-false question doesn't allow you to be ``partially correct'',
there's nothing in between 5/10 and 6/10. Table~\ref{tbl-tab2-4}
summarises the relationship between the scales of measurement and the
discrete/continuity distinction. Cells with a tick mark correspond to
things that are possible. I'm trying to hammer this point home, because
(a) some textbooks get this wrong, and (b) people very often say things
like ``discrete variable'' when they mean ``nominal scale variable''.
It's very unfortunate.

\hypertarget{some-complexities}{%
\subsection{Some complexities}\label{some-complexities}}

Okay, I know you're going to be shocked to hear this, but the real world
is much messier than this little classification scheme suggests. Very
few variables in real life actually fall into these nice neat
categories, so you need to be kind of careful not to treat the scales of
measurement as if they were hard and fast rules. It doesn't work like
that. They're guidelines, intended to help you think about the
situations in which you should treat different variables differently.
Nothing more.

So let's take a classic example, maybe \emph{the} classic example, of a
psychological measurement tool: the \textbf{Likert scale}. The humble
Likert scale is the bread and butter tool of all survey design. You
yourself have filled out hundreds, maybe thousands, of them and odds are
you've even used one yourself. Suppose we have a survey question that
looks like this:

\begin{quote}
Which of the following best describes your opinion of the statement that
``all pirates are freaking awesome''?
\end{quote}

and then the options presented to the participant are these:

\begin{enumerate}
\def\labelenumi{\arabic{enumi}.}
\tightlist
\item
  Strongly disagree
\item
  Disagree
\item
  Neither agree nor disagree
\item
  Agree
\item
  Strongly agree
\end{enumerate}

This set of items is an example of a 5-point Likert scale, in which
people are asked to choose among one of several (in this case 5) clearly
ordered possibilities, generally with a verbal descriptor given in each
case. However, it's not necessary that all items are explicitly
described. This is a perfectly good example of a 5-point Likert scale
too:

\begin{enumerate}
\def\labelenumi{\arabic{enumi}.}
\tightlist
\item
  Strongly disagree
\item
\item
\item
\item
  Strongly agree
\end{enumerate}

Likert scales are very handy, if somewhat limited, tools. The question
is what kind of variable are they? They're obviously discrete, since you
can't give a response of 2.5. They're obviously not nominal scale, since
the items are ordered; and they're not ratio scale either, since there's
no natural zero.

But are they ordinal scale or interval scale? One argument says that we
can't really prove that the difference between ``strongly agree'' and
``agree'' is of the same size as the difference between ``agree'' and
``neither agree nor disagree''. In fact, in everyday life it's pretty
obvious that they're not the same at all. So this suggests that we ought
to treat Likert scales as ordinal variables. On the other hand, in
practice most participants do seem to take the whole ``on a scale from 1
to 5'' part fairly seriously, and they tend to act as if the differences
between the five response options were fairly similar to one another. As
a consequence, a lot of researchers treat Likert scale data as interval
scale.\footnote{Ah, psychology\ldots{} never an easy answer to anything!}
It's not interval scale, but in practice it's close enough that we
usually think of it as being \textbf{quasi-interval scale}.

\hypertarget{sec-Assessing-the-reliability-of-a-measurement}{%
\section{Assessing the reliability of a
measurement}\label{sec-Assessing-the-reliability-of-a-measurement}}

At this point we've thought a little bit about how to operationalise a
theoretical construct and thereby create a psychological measure. And
we've seen that by applying psychological measures we end up with
variables, which can come in many different types. At this point, we
should start discussing the obvious question: is the measurement any
good? We'll do this in terms of two related ideas: \emph{reliability}
and \emph{validity}. Put simply, the \textbf{reliability} of a measure
tells you how precisely you are measuring something, whereas the
\textbf{validity} of a measure tells you how accurate the measure is. In
this section we'll talk about reliability; we'll talk about validity in
the section on
\protect\hyperlink{assessing-the-validity-of-a-study}{Assessing the
validity of a study}.

Reliability is actually a very simple concept. It refers to the
repeatability or consistency of your measurement. The measurement of my
weight by means of a ``bathroom scale'' is very reliable. If I step on
and off the scales over and over again, it'll keep giving me the same
answer. Measuring my intelligence by means of ``asking my mum'' is very
unreliable. Some days she tells me I'm a bit thick, and other days she
tells me I'm a complete idiot. Notice that this concept of reliability
is different to the question of whether the measurements are correct
(the correctness of a measurement relates to it's validity). If I'm
holding a sack of potatoes when I step on and off the bathroom scales
the measurement will still be reliable: it will always give me the same
answer. However, this highly reliable answer doesn't match up to my true
weight at all, therefore it's wrong. In technical terms, this is a
reliable but invalid measurement. Similarly, whilst my mum's estimate of
my intelligence is a bit unreliable, she might be right. Maybe I'm just
not too bright, and so while her estimate of my intelligence fluctuates
pretty wildly from day to day, it's basically right. That would be an
unreliable but valid measure. Of course, if my mum's estimates are too
unreliable it's going to be very hard to figure out which one of her
many claims about my intelligence is actually the right one. To some
extent, then, a very unreliable measure tends to end up being invalid
for practical purposes; so much so that many people would say that
reliability is necessary (but not sufficient) to ensure validity.

Okay, now that we're clear on the distinction between reliability and
validity, let's have a think about the different ways in which we might
measure reliability:

\begin{itemize}
\tightlist
\item
  \textbf{Test-retest reliability}. This relates to consistency over
  time. If we repeat the measurement at a later date do we get the same
  answer?
\item
  \textbf{Inter-rater reliability}. This relates to consistency across
  people. If someone else repeats the measurement (e.g., someone else
  rates my intelligence) will they produce the same answer?
\item
  \textbf{Parallel forms reliability}. This relates to consistency
  across theoretically-equivalent measurements. If I use a different set
  of bathroom scales to measure my weight does it give the same answer?
\item
  \textbf{Internal consistency reliability}. If a measurement is
  constructed from lots of different parts that perform similar
  functions (e.g., a personality questionnaire result is added up across
  several questions) do the individual parts tend to give similar
  answers. We'll look at this particular form of reliability later in
  the book, in \textbf{?@sec-Internal-consistency-reliability-analysis}.
\end{itemize}

Not all measurements need to possess all forms of reliability. For
instance, educational assessment can be thought of as a form of
measurement. One of the subjects that I teach, \emph{Computational
Cognitive Science}, has an assessment structure that has a research
component and an exam component (plus other things). The exam component
is \emph{intended} to measure something different from the research
component, so the assessment as a whole has low internal consistency.
However, within the exam there are several questions that are intended
to (approximately) measure the same things, and those tend to produce
similar outcomes. So the exam on its own has a fairly high internal
consistency. Which is as it should be. You should only demand
reliability in those situations where you want to be measuring the same
thing!

\hypertarget{the-role-of-variables-predictors-and-outcomes}{%
\section{The ``role'' of variables: predictors and
outcomes}\label{the-role-of-variables-predictors-and-outcomes}}

I've got one last piece of terminology that I need to explain to you
before moving away from variables. Normally, when we do some research we
end up with lots of different variables. Then, when we analyse our data,
we usually try to explain some of the variables in terms of some of the
other variables. It's important to keep the two roles ``thing doing the
explaining'' and ``thing being explained'' distinct. So let's be clear
about this now. First, we might as well get used to the idea of using
mathematical symbols to describe variables, since it's going to happen
over and over again. Let's denote the ``to be explained'' variable
\(Y\), and denote the variables ``doing the explaining'' as
\(X_1 , X_2\), etc.

When we are doing an analysis we have different names for \(X\) and
\(Y\), since they play different roles in the analysis. The classical
names for these roles are \textbf{independent variable} (IV) and
\textbf{dependent variable} (DV). The IV is the variable that you use to
do the explaining (i.e., \(X\)) and the DV is the variable being
explained (i.e., \(Y\)). The logic behind these names goes like this: if
there really is a relationship between \(X\) and \(Y\) then we can say
that \(Y\)depends on \(X\), and if we have designed our study
``properly'' then \(X\) isn't dependent on anything else. However, I
personally find those names horrible. They're hard to remember and
they're highly misleading because (a) the IV is never actually
``independent of everything else'', and (b) if there's no relationship
then the DV doesn't actually depend on the IV. And in fact, because I'm
not the only person who thinks that IV and DV are just awful names,
there are a number of alternatives that I find more appealing. The terms
that I'll use in this book are \textbf{predictors} and
\textbf{outcomes}. The idea here is that what you're trying to do is use
\(X\) (the predictors) to make guesses about \(Y\) (the
outcomes).\footnote{Annoyingly though, there's a lot of different names
  used out there. I won't list all of them -- there would be no point in
  doing that -- other than to note that ``response variable'' is
  sometimes used where I've used ``outcome''. Sigh. This sort of
  terminological confusion is very common, I'm afraid.} This is
summarised in Table~\ref{tbl-tab2-5}.

\hypertarget{tbl-tab2-5}{}
 
  \providecommand{\huxb}[2]{\arrayrulecolor[RGB]{#1}\global\arrayrulewidth=#2pt}
  \providecommand{\huxvb}[2]{\color[RGB]{#1}\vrule width #2pt}
  \providecommand{\huxtpad}[1]{\rule{0pt}{#1}}
  \providecommand{\huxbpad}[1]{\rule[-#1]{0pt}{#1}}

\begin{table}[ht]
\caption{\label{tbl-tab2-5}Variable distinctions }\tabularnewline

\begin{centerbox}
\begin{threeparttable}
\setlength{\tabcolsep}{0pt}
\begin{tabularx}{0.9\textwidth}{p{0.3\textwidth} p{0.3\textwidth} p{0.3\textwidth}}


\hhline{>{\huxb{0, 0, 0}{0.4}}->{\huxb{0, 0, 0}{0.4}}->{\huxb{0, 0, 0}{0.4}}-}
\arrayrulecolor{black}

\multicolumn{1}{!{\huxvb{0, 0, 0}{0}}p{0.3\textwidth}!{\huxvb{0, 0, 0}{0}}}{\hspace{0pt}\parbox[b]{0.3\textwidth-0pt-12pt}{\huxtpad{2pt + 1em}\centering \textbf{role of the variable}\huxbpad{2pt}}} &
\multicolumn{1}{p{0.3\textwidth}!{\huxvb{0, 0, 0}{0}}}{\hspace{12pt}\parbox[b]{0.3\textwidth-12pt-12pt}{\huxtpad{2pt + 1em}\centering \textbf{classical name}\huxbpad{2pt}}} &
\multicolumn{1}{p{0.3\textwidth}!{\huxvb{0, 0, 0}{0}}}{\hspace{12pt}\parbox[b]{0.3\textwidth-12pt-0pt}{\huxtpad{2pt + 1em}\centering \textbf{modern name}\huxbpad{2pt}}} \tabularnewline[-0.5pt]


\hhline{>{\huxb{0, 0, 0}{0.4}}->{\huxb{0, 0, 0}{0.4}}->{\huxb{0, 0, 0}{0.4}}-}
\arrayrulecolor{black}

\multicolumn{1}{!{\huxvb{0, 0, 0}{0}}p{0.3\textwidth}!{\huxvb{0, 0, 0}{0}}}{\hspace{0pt}\parbox[b]{0.3\textwidth-0pt-12pt}{\huxtpad{2pt + 1em}\centering \(\text{``}\)to be explained\(\text{''}\)\huxbpad{2pt}}} &
\multicolumn{1}{p{0.3\textwidth}!{\huxvb{0, 0, 0}{0}}}{\hspace{12pt}\parbox[b]{0.3\textwidth-12pt-12pt}{\huxtpad{2pt + 1em}\centering dependent variable (DV)\huxbpad{2pt}}} &
\multicolumn{1}{p{0.3\textwidth}!{\huxvb{0, 0, 0}{0}}}{\hspace{12pt}\parbox[b]{0.3\textwidth-12pt-0pt}{\huxtpad{2pt + 1em}\centering outcome\huxbpad{2pt}}} \tabularnewline[-0.5pt]


\hhline{}
\arrayrulecolor{black}

\multicolumn{1}{!{\huxvb{0, 0, 0}{0}}p{0.3\textwidth}!{\huxvb{0, 0, 0}{0}}}{\hspace{0pt}\parbox[b]{0.3\textwidth-0pt-12pt}{\huxtpad{2pt + 1em}\centering \(\text{``}\)to do the explaining\(\text{''}\)\huxbpad{2pt}}} &
\multicolumn{1}{p{0.3\textwidth}!{\huxvb{0, 0, 0}{0}}}{\hspace{12pt}\parbox[b]{0.3\textwidth-12pt-12pt}{\huxtpad{2pt + 1em}\centering independent variable (IV)\huxbpad{2pt}}} &
\multicolumn{1}{p{0.3\textwidth}!{\huxvb{0, 0, 0}{0}}}{\hspace{12pt}\parbox[b]{0.3\textwidth-12pt-0pt}{\huxtpad{2pt + 1em}\centering predictor\huxbpad{2pt}}} \tabularnewline[-0.5pt]


\hhline{>{\huxb{0, 0, 0}{0.4}}->{\huxb{0, 0, 0}{0.4}}->{\huxb{0, 0, 0}{0.4}}-}
\arrayrulecolor{black}
\end{tabularx} 

\end{threeparttable}\par\end{centerbox}

\end{table}
 

\hypertarget{experimental-and-non-experimental-research}{%
\section{Experimental and non-experimental
research}\label{experimental-and-non-experimental-research}}

One of the big distinctions that you should be aware of is the
distinction between ``experimental research'' and ``non-experimental
research''. When we make this distinction, what we're really talking
about is the degree of control that the researcher exercises over the
people and events in the study.

\hypertarget{experimental-research}{%
\subsection{Experimental research}\label{experimental-research}}

The key feature of \textbf{experimental research} is that the researcher
controls all aspects of the study, especially what participants
experience during the study. In particular, the researcher manipulates
or varies the predictor variables (IVs) but allows the outcome variable
(DV) to vary naturally. The idea here is to deliberately vary the
predictors (IVs) to see if they have any causal effects on the outcomes.
Moreover, in order to ensure that there's no possibility that something
other than the predictor variables is causing the outcomes, everything
else is kept constant or is in some other way ``balanced'', to ensure
that they have no effect on the results. In practice, it's almost
impossible to \emph{think} of everything else that might have an
influence on the outcome of an experiment, much less keep it constant.
The standard solution to this is \textbf{randomisation}. That is, we
randomly assign people to different groups, and then give each group a
different treatment (i.e., assign them different values of the predictor
variables). We'll talk more about randomisation later, but for now it's
enough to say that what randomisation does is minimise (but not
eliminate) the possibility that there are any systematic difference
between groups.

Let's consider a very simple, completely unrealistic and grossly
unethical example. Suppose you wanted to find out if smoking causes lung
cancer. One way to do this would be to find people who smoke and people
who don't smoke and look to see if smokers have a higher rate of lung
cancer. This is \emph{not} a proper experiment, since the researcher
doesn't have a lot of control over who is and isn't a smoker. And this
really matters. For instance, it might be that people who choose to
smoke cigarettes also tend to have poor diets, or maybe they tend to
work in asbestos mines, or whatever. The point here is that the groups
(smokers and non-smokers) actually differ on lots of things, not just
smoking. So it might be that the higher incidence of lung cancer among
smokers is caused by something else, and not by smoking per se. In
technical terms these other things (e.g., diet) are called
``confounders'', and we'll talk about those in just a moment.

In the meantime, let's consider what a proper experiment might look
like. Recall that our concern was that smokers and non-smokers might
differ in lots of ways. The solution, as long as you have no ethics, is
to control who smokes and who doesn't. Specifically, if we randomly
divide young non-smokers into two groups and force half of them to
become smokers, then it's very unlikely that the groups will differ in
any respect other than the fact that half of them smoke. That way, if
our smoking group gets cancer at a higher rate than the non-smoking
group, we can feel pretty confident that (a) smoking does cause cancer
and (b) we're murderers.

\hypertarget{non-experimental-research}{%
\subsection{Non-experimental research}\label{non-experimental-research}}

\textbf{Non-experimental research} is a broad term that covers ``any
study in which the researcher doesn't have as much control as they do in
an experiment''. Obviously, control is something that scientists like to
have, but as the previous example illustrates there are lots of
situations in which you can't or shouldn't try to obtain that control.
Since it's grossly unethical (and almost certainly criminal) to force
people to smoke in order to find out if they get cancer, this is a good
example of a situation in which you really shouldn't try to obtain
experimental control. But there are other reasons too. Even leaving
aside the ethical issues, our ``smoking experiment'' does have a few
other issues. For instance, when I suggested that we ``force'' half of
the people to become smokers, I was talking about \emph{starting} with a
sample of non-smokers, and then forcing them to become smokers. While
this sounds like the kind of solid, evil experimental design that a mad
scientist would love, it might not be a very sound way of investigating
the effect in the real world. For instance, suppose that smoking only
causes lung cancer when people have poor diets, and suppose also that
people who normally smoke do tend to have poor diets. However, since the
``smokers'' in our experiment aren't ``natural'' smokers (i.e., we
forced non-smokers to become smokers, but they didn't take on all of the
other normal, real-life characteristics that smokers might tend to
possess) they probably have better diets. As such, in this silly example
they wouldn't get lung cancer and our experiment will fail, because it
violates the structure of the ``natural'' world (the technical name for
this is an ``artefactual'' result).

One distinction worth making between two types of non-experimental
research is the difference between \textbf{quasi-experimental research}
and \textbf{case studies}. The example I discussed earlier, in which we
wanted to examine incidence of lung cancer among smokers and non-smokers
without trying to control who smokes and who doesn't, is a
quasi-experimental design. That is, it's the same as an experiment but
we don't control the predictors (IVs). We can still use statistics to
analyse the results, but we have to be a lot more careful and
circumspect.

The alternative approach, case studies, aims to provide a very detailed
description of one or a few instances. In general, you can't use
statistics to analyse the results of case studies and it's usually very
hard to draw any general conclusions about ``people in general'' from a
few isolated examples. However, case studies are very useful in some
situations. Firstly, there are situations where you don't have any
alternative. Neuropsychology has this issue a lot. Sometimes, you just
can't find a lot of people with brain damage in a specific brain area,
so the only thing you can do is describe those cases that you do have in
as much detail and with as much care as you can. However, there's also
some genuine advantages to case studies. Because you don't have as many
people to study you have the ability to invest lots of time and effort
trying to understand the specific factors at play in each case. This is
a very valuable thing to do. As a consequence, case studies can
complement the more statistically-oriented approaches that you see in
experimental and quasi-experimental designs. We won't talk much about
case studies in this book, but they are nevertheless very valuable
tools!

\hypertarget{assessing-the-validity-of-a-study}{%
\section{Assessing the validity of a
study}\label{assessing-the-validity-of-a-study}}

More than any other thing, a scientist wants their research to be
``valid''. The conceptual idea behind \textbf{validity} is very simple.
Can you trust the results of your study? If not, the study is invalid.
However, whilst it's easy to state, in practice it's much harder to
check validity than it is to check reliability. And in all honesty,
there's no precise, clearly agreed upon notion of what validity actually
is. In fact, there are lots of different kinds of validity, each of
which raises its own issues. And not all forms of validity are relevant
to all studies. I'm going to talk about five different types of
validity:

\begin{itemize}
\tightlist
\item
  Internal validity.
\item
  External validity.
\item
  Construct validity.
\item
  Face validity.
\item
  Ecological validity.
\end{itemize}

First, a quick guide as to what matters here. (1) Internal and external
validity are the most important, since they tie directly to the
fundamental question of whether your study really works. (2) Construct
validity asks whether you're measuring what you think you are. (3) Face
validity isn't terribly important except insofar as you care about
``appearances''. (4) Ecological validity is a special case of face
validity that corresponds to a kind of appearance that you might care
about a lot.

\hypertarget{internal-validity}{%
\subsection{Internal validity}\label{internal-validity}}

\textbf{Internal validity} refers to the extent to which you are able to
draw the correct conclusions about the causal relationships between
variables. It's called ``internal'' because it refers to the
relationships between things ``inside'' the study. Let's illustrate the
concept with a simple example. Suppose you're interested in finding out
whether a university education makes you write better. To do so, you get
a group of first year students, ask them to write a 1000 word essay, and
count the number of spelling and grammatical errors they make. Then you
find some third-year students, who obviously have had more of a
university education than the first-years, and repeat the exercise. And
let's suppose it turns out that the third-year students produce fewer
errors. And so you conclude that a university education improves writing
skills. Right? Except that the big problem with this experiment is that
the third-year students are older and they've had more experience with
writing things. So it's hard to know for sure what the causal
relationship is. Do older people write better? Or people who have had
more writing experience? Or people who have had more education? Which of
the above is the true cause of the superior performance of the
third-years? Age? Experience? Education? You can't tell. This is an
example of a failure of internal validity, because your study doesn't
properly tease apart the causal relationships between the different
variables.

\hypertarget{external-validity}{%
\subsection{External validity}\label{external-validity}}

\textbf{External validity} relates to the \textbf{generalisability} or
\textbf{applicability} of your findings. That is, to what extent do you
expect to see the same pattern of results in ``real life'' as you saw in
your study. To put it a bit more precisely, any study that you do in
psychology will involve a fairly specific set of questions or tasks,
will occur in a specific environment, and will involve participants that
are drawn from a particular subgroup (disappointingly often it is
college students!). So, if it turns out that the results don't actually
generalise or apply to people and situations beyond the ones that you
studied, then what you've got is a lack of external validity.

The classic example of this issue is the fact that a very large
proportion of studies in psychology will use undergraduate psychology
students as the participants. Obviously, however, the researchers don't
care \emph{only} about psychology students. They care about people in
general. Given that, a study that uses only psychology students as
participants always carries a risk of lacking external validity. That
is, if there's something ``special'' about psychology students that
makes them different to the general population in some relevant respect,
then we may start worrying about a lack of external validity.

That said, it is absolutely critical to realise that a study that uses
only psychology students does not necessarily have a problem with
external validity. I'll talk about this again later, but it's such a
common mistake that I'm going to mention it here. The external validity
of a study is threatened by the choice of population if (a) the
population from which you sample your participants is very narrow (e.g.,
psychology students), and (b) the narrow population that you sampled
from is systematically different from the general population in some
respect that is relevant to the \emph{psychological phenomenon that you
intend to study}. The italicised part is the bit that lots of people
forget. It is true that psychology undergraduates differ from the
general population in lots of ways, and so a study that uses only
psychology students may have problems with external validity. However,
if those differences aren't very relevant to the phenomenon that you're
studying, then there's nothing to worry about. To make this a bit more
concrete here are two extreme examples:

\begin{itemize}
\tightlist
\item
  You want to measure ``attitudes of the general public towards
  psychotherapy'', but all of your participants are psychology students.
  This study would almost certainly have a problem with external
  validity.
\item
  You want to measure the effectiveness of a visual illusion, and your
  participants are all psychology students. This study is unlikely to
  have a problem with external validity.
\end{itemize}

Having just spent the last couple of paragraphs focusing on the choice
of participants, since that's a big issue that everyone tends to worry
most about, it's worth remembering that external validity is a broader
concept. The following are also examples of things that might pose a
threat to external validity, depending on what kind of study you're
doing:

\begin{itemize}
\tightlist
\item
  People might answer a ``psychology questionnaire'' in a manner that
  doesn't reflect what they would do in real life.
\item
  Your lab experiment on (say) ``human learning'' has a different
  structure to the learning problems people face in real life.
\end{itemize}

\hypertarget{construct-validity}{%
\subsection{Construct validity}\label{construct-validity}}

\textbf{Construct validity} is basically a question of whether you're
measuring what you want to be measuring. A measurement has good
construct validity if it is actually measuring the correct theoretical
construct, and bad construct validity if it doesn't. To give a very
simple (if ridiculous) example, suppose I'm trying to investigate the
rates with which university students cheat on their exams. And the way I
attempt to measure it is by asking the cheating students to stand up in
the lecture theatre so that I can count them. When I do this with a
class of 300 students 0 people claim to be cheaters. So I therefore
conclude that the proportion of cheaters in my class is 0\%. Clearly
this is a bit ridiculous. But the point here is not that this is a very
deep methodological example, but rather to explain what construct
validity is. The problem with my measure is that while I'm trying to
measure ``the proportion of people who cheat'' what I'm actually
measuring is ``the proportion of people stupid enough to own up to
cheating, or bloody minded enough to pretend that they do''. Obviously,
these aren't the same thing! So my study has gone wrong, because my
measurement has very poor construct validity.

\hypertarget{face-validity}{%
\subsection{Face validity}\label{face-validity}}

\textbf{Face validity} simply refers to whether or not a measure ``looks
like'' it's doing what it's supposed to, nothing more. If I design a
test of intelligence, and people look at it and they say ``no, that test
doesn't measure intelligence'', then the measure lacks face validity.
It's as simple as that. Obviously, face validity isn't very important
from a pure scientific perspective. After all, what we care about is
whether or not the measure \emph{actually} does what it's supposed to
do, not whether it \emph{looks like} it does what it's supposed to do.
As a consequence, we generally don't care very much about face validity.
That said, the concept of face validity serves three useful pragmatic
purposes:

\begin{itemize}
\tightlist
\item
  Sometimes, an experienced scientist will have a ``hunch'' that a
  particular measure won't work. While these sorts of hunches have no
  strict evidentiary value, it's often worth paying attention to them.
  Because often times people have knowledge that they can't quite
  verbalise, there might be something to worry about even if you can't
  quite say why. In other words, when someone you trust criticises the
  face validity of your study, it's worth taking the time to think more
  carefully about your design to see if you can think of reasons why it
  might go awry. Mind you, if you don't find any reason for concern,
  then you should probably not worry. After all, face validity really
  doesn't matter very much.
\item
  Often (very often), completely uninformed people will also have a
  ``hunch'' that your research is crap. And they'll criticise it on the
  internet or something. On close inspection you may notice that these
  criticisms are actually focused entirely on how the study ``looks'',
  but not on anything deeper. The concept of face validity is useful for
  gently explaining to people that they need to substantiate their
  arguments further.
\item
  Expanding on the last point, if the beliefs of untrained people are
  critical (e.g., this is often the case for applied research where you
  actually want to convince policy makers of something or other) then
  you have to care about face validity. Simply because, whether you like
  it or not, a lot of people will use face validity as a proxy for real
  validity. If you want the government to change a law on scientific
  psychological grounds, then it won't matter how good your studies
  ``really'' are. If they lack face validity you'll find that
  politicians ignore you. Of course, it's somewhat unfair that policy
  often depends more on appearance than fact, but that's how things go.
\end{itemize}

\hypertarget{ecological-validity}{%
\subsection{Ecological validity}\label{ecological-validity}}

\textbf{Ecological validity} is a different notion of validity, which is
similar to external validity, but less important. The idea is that, in
order to be ecologically valid, the entire set up of the study should
closely approximate the real-world scenario that is being investigated.
In a sense, ecological validity is a kind of face validity. It relates
mostly to whether the study ``looks'' right, but with a bit more rigour
to it. To be ecologically valid the study has to look right in a fairly
specific way. The idea behind it is the intuition that a study that is
ecologically valid is more likely to be externally valid. It's no
guarantee, of course. But the nice thing about ecological validity is
that it's much easier to check whether a study is ecologically valid
than it is to check whether a study is externally valid. A simple
example would be eyewitness identification studies. Most of these
studies tend to be done in a university setting, often with a fairly
simple array of faces to look at, rather than a line up. The length of
time between seeing the ``criminal'' and being asked to identify the
suspect in the ``line up'' is usually shorter. The ``crime'' isn't real
so there's no chance of the witness being scared, and there are no
police officers present so there's not as much chance of feeling
pressured. These things all mean that the study definitely lacks
ecological validity. They might (but might not) mean that it also lacks
external validity.

\hypertarget{confounders-artefacts-and-other-threats-to-validity}{%
\section{Confounders, artefacts and other threats to
validity}\label{confounders-artefacts-and-other-threats-to-validity}}

If we look at the issue of validity in the most general fashion the two
biggest worries that we have are \emph{confounders} and
\emph{artefacts}. These two terms are defined in the following way:

\begin{itemize}
\tightlist
\item
  \textbf{Confounder:} A confounder is an additional, often unmeasured
  variable\footnote{The reason why I say that it's unmeasured is that if
    you have measured it, then you can use some fancy statistical tricks
    to deal with the confounder. Because of the existence of these
    statistical solutions to the problem of confounders, we often refer
    to a confounder that we have measured and dealt with as a covariate.
    Dealing with covariates is a more advanced topic, but I thought I'd
    mention it in passing since it's kind of comforting to at least know
    that this stuff exists.} that turns out to be related to both the
  predictors and the outcome. The existence of confounders threatens the
  internal validity of the study because you can't tell whether the
  predictor causes the outcome, or if the confounding variable causes
  it.
\item
  \textbf{Artefact:} A result is said to be ``artefactual'' if it only
  holds in the special situation that you happened to test in your
  study. The possibility that your result is an artefact poses a threat
  to your external validity, because it raises the possibility that you
  can't generalise or apply your results to the actual population that
  you care about.
\end{itemize}

As a general rule confounders are a bigger concern for non-experimental
studies, precisely because they're not proper experiments. By
definition, you're leaving lots of things uncontrolled, so there's a lot
of scope for confounders being present in your study. Experimental
research tends to be much less vulnerable to confounders. The more
control you have over what happens during the study, the more you can
prevent confounders from affecting the results. With random allocation,
for example, confounders are distributed randomly, and evenly, between
different groups.

However, there are always swings and roundabouts and when we start
thinking about artefacts rather than confounders the shoe is very firmly
on the other foot. For the most part, artefactual results tend to be
more of a concern for experimental studies than for non-experimental
studies. To see this, it helps to realise that the reason that a lot of
studies are non-experimental is precisely because what the researcher is
trying to do is examine human behaviour in a more naturalistic context.
By working in a more real-world context you lose experimental control
(making yourself vulnerable to confounders), but because you tend to be
studying human psychology ``in the wild'' you reduce the chances of
getting an artefactual result. Or, to put it another way, when you take
psychology out of the wild and bring it into the lab (which we usually
have to do to gain our experimental control), you always run the risk of
accidentally studying something different to what you wanted to study.

Be warned though. The above is a rough guide only. It's absolutely
possible to have confounders in an experiment, and to get artefactual
results with non-experimental studies. This can happen for all sorts of
reasons, not least of which is experimenter or researcher error. In
practice, it's really hard to think everything through ahead of time and
even very good researchers make mistakes.

Although there's a sense in which almost any threat to validity can be
characterised as a confounder or an artefact, they're pretty vague
concepts. So let's have a look at some of the most common examples.

\hypertarget{history-effects}{%
\subsection{History effects}\label{history-effects}}

\textbf{History effects} refer to the possibility that specific events
may occur during the study that might influence the outcome measure. For
instance, something might happen in between a pretest and a post-test.
Or in-between testing participant 23 and participant 24. Alternatively,
it might be that you're looking at a paper from an older study that was
perfectly valid for its time, but the world has changed enough since
then that the conclusions are no longer trustworthy. Examples of things
that would count as history effects are:

\begin{itemize}
\item
  You're interested in how people think about risk and uncertainty. You
  started your data collection in December 2010. But finding
  participants and collecting data takes time, so you're still finding
  new people in February 2011. Unfortunately for you (and even more
  unfortunately for others), the Queensland floods occurred in January
  2011 causing billions of dollars of damage and killing many people.
  Not surprisingly, the people tested in February 2011 express quite
  different beliefs about handling risk than the people tested in
  December 2010. Which (if any) of these reflects the ``true'' beliefs
  of participants? I think the answer is probably both. The Queensland
  floods genuinely changed the beliefs of the Australian public, though
  possibly only temporarily. The key thing here is that the ``history''
  of the people tested in February is quite different to people tested
  in December.
\item
  You're testing the psychological effects of a new anti-anxiety drug.
  So what you do is measure anxiety before administering the drug (e.g.,
  by self-report, and taking physiological measures). Then you
  administer the drug, and afterwards you take the same measures. In the
  interim however, because your lab is in Los Angeles, there's an
  earthquake which increases the anxiety of the participants.
\end{itemize}

\hypertarget{maturation-effects}{%
\subsection{Maturation effects}\label{maturation-effects}}

As with history effects, \textbf{maturational effects} are fundamentally
about change over time. However, maturation effects aren't in response
to specific events. Rather, they relate to how people change on their
own over time. We get older, we get tired, we get bored, etc. Some
examples of maturation effects are:

\begin{itemize}
\tightlist
\item
  When doing developmental psychology research you need to be aware that
  children grow up quite rapidly. So, suppose that you want to find out
  whether some educational trick helps with vocabulary size among 3 year
  olds. One thing that you need to be aware of is that the vocabulary
  size of children that age is growing at an incredible rate (multiple
  words per day) all on its own. If you design your study without taking
  this maturational effect into account, then you won't be able to tell
  if your educational trick works.
\item
  When running a very long experiment in the lab (say, something that
  lasts for three hours) it's very likely that people will begin to get
  bored and tired, and that this maturational effect will cause
  performance to decline regardless of anything else going on in the
  experiment.
\end{itemize}

\hypertarget{repeated-testing-effects}{%
\subsection{Repeated testing effects}\label{repeated-testing-effects}}

An important type of history effect is the effect of \textbf{repeated
testing}. Suppose I want to take two measurements of some psychological
construct (e.g., anxiety). One thing I might be worried about is if the
first measurement has an effect on the second measurement. In other
words, this is a history effect in which the ``event'' that influences
the second measurement is the first measurement itself! This is not at
all uncommon. Examples of this include:

\begin{itemize}
\tightlist
\item
  Learning and practice: e.g., ``intelligence'' at time 2 might appear
  to go up relative to time 1 because participants learned the general
  rules of how to solve ``intelligence-test-style'' questions during the
  first testing session.
\item
  Familiarity with the testing situation: e.g., if people are nervous at
  time 1, this might make performance go down. But after sitting through
  the first testing situation they might calm down a lot precisely
  because they've seen what the testing looks like.
\item
  Auxiliary changes caused by testing: e.g., if a questionnaire
  assessing mood is boring then mood rating at measurement time 2 is
  more likely to be ``bored'' precisely because of the boring
  measurement made at time 1.
\end{itemize}

\hypertarget{selection-bias}{%
\subsection{Selection bias}\label{selection-bias}}

\textbf{Selection bias} is a pretty broad term. Suppose that you're
running an experiment with two groups of participants where each group
gets a different ``treatment'', and you want to see if the different
treatments lead to different outcomes. However, suppose that, despite
your best efforts, you've ended up with a gender imbalance across groups
(say, group A has 80\% females and group B has 50\% females). It might
sound like this could never happen but, trust me, it can. This is an
example of a selection bias, in which the people ``selected into'' the
two groups have different characteristics. If any of those
characteristics turns out to be relevant (say, your treatment works
better on females than males) then you're in a lot of trouble.

\hypertarget{differential-attrition}{%
\subsection{Differential attrition}\label{differential-attrition}}

When thinking about the effects of attrition, it is sometimes helpful to
distinguish between two different types. The first is
\textbf{homogeneous attrition}, in which the attrition effect is the
same for all groups, treatments or conditions. In the example I gave
above, the attrition would be homogeneous if (and only if) the easily
bored participants are dropping out of all of the conditions in my
experiment at about the same rate. In general, the main effect of
homogeneous attrition is likely to be that it makes your sample
unrepresentative. As such, the biggest worry that you'll have is that
the generalisability of the results decreases. In other words, you lose
external validity.

The second type of attrition is \textbf{heterogeneous attrition}, in
which the attrition effect is different for different groups. More often
called \textbf{differential attrition}, this is a kind of selection bias
that is caused by the study itself. Suppose that, for the first time
ever in the history of psychology, I manage to find the perfectly
balanced and representative sample of people. I start running ``Dani's
incredibly long and tedious experiment'' on my perfect sample but then,
because my study is incredibly long and tedious, lots of people start
dropping out. I can't stop this. Participants absolutely have the right
to stop doing any experiment, any time, for whatever reason they feel
like, and as researchers we are morally (and professionally) obliged to
remind people that they do have this right. So, suppose that ``Dani's
incredibly long and tedious experiment'' has a very high drop out rate.
What do you suppose the odds are that this drop out is random? Answer:
zero. Almost certainly the people who remain are more conscientious,
more tolerant of boredom, etc., than those who leave. To the extent that
(say) conscientiousness is relevant to the psychological phenomenon that
I care about, this attrition can decrease the validity of my results.

Here's another example. Suppose I design my experiment with two
conditions. In the ``treatment'' condition, the experimenter insults the
participant and then gives them a questionnaire designed to measure
obedience. In the ``control'' condition, the experimenter engages in a
bit of pointless chitchat and then gives them the questionnaire. Leaving
aside the questionable scientific merits and dubious ethics of such a
study, let's have a think about what might go wrong here. As a general
rule, when someone insults me to my face I tend to get much less
co-operative. So, there's a pretty good chance that a lot more people
are going to drop out of the treatment condition than the control
condition. And this drop out isn't going to be random. The people most
likely to drop out would probably be the people who don't care all that
much about the importance of obediently sitting through the experiment.
Since the most bloody minded and disobedient people all left the
treatment group but not the control group, we've introduced a
confounder: the people who actually took the questionnaire in the
treatment group were already more likely to be dutiful and obedient than
the people in the control group. In short, in this study insulting
people doesn't make them more obedient. It makes the more disobedient
people leave the experiment! The internal validity of this experiment is
completely shot.

\hypertarget{non-response-bias}{%
\subsection{Non-response bias}\label{non-response-bias}}

\textbf{Non-response bias} is closely related to selection bias and to
differential attrition. The simplest version of the problem goes like
this. You mail out a survey to 1000 people but only 300 of them reply.
The 300 people who replied are almost certainly not a random subsample.
People who respond to surveys are systematically different to people who
don't. This introduces a problem when trying to generalise from those
300 people who replied to the population at large, since you now have a
very non-random sample. The issue of non-response bias is more general
than this, though. Among the (say) 300 people that did respond to the
survey, you might find that not everyone answers every question. If
(say) 80 people chose not to answer one of your questions, does this
introduce problems? As always, the answer is maybe. If the question that
wasn't answered was on the last page of the questionnaire, and those 80
surveys were returned with the last page missing, there's a good chance
that the missing data isn't a big deal; probably the pages just fell
off. However, if the question that 80 people didn't answer was the most
confrontational or invasive personal question in the questionnaire, then
almost certainly you've got a problem. In essence, what you're dealing
with here is what's called the problem of \textbf{missing data}. If the
data that is missing was ``lost'' randomly, then it's not a big problem.
If it's missing systematically, then it can be a big problem.

\hypertarget{regression-to-the-mean}{%
\subsection{Regression to the mean}\label{regression-to-the-mean}}

\textbf{Regression to the mean} refers to any situation where you select
data based on an extreme value on some measure. Because the variable has
natural variation it almost certainly means that when you take a
subsequent measurement the later measurement will be less extreme than
the first one, purely by chance.

Here's an example. Suppose I'm interested in whether a psychology
education has an adverse effect on very smart kids. To do this, I find
the 20 Psychology I students with the best high school grades and look
at how well they're doing at university. It turns out that they're doing
a lot better than average, but they're not topping the class at
university even though they did top their classes at high school. What's
going on? The natural first thought is that this must mean that the
psychology classes must be having an adverse effect on those students.
However, while that might very well be the explanation, it's more likely
that what you're seeing is an example of ``regression to the mean''. To
see how it works, let's take a moment to think about what is required to
get the best mark in a class, regardless of whether that class be at
high school or at university. When you've got a big class there are
going to be lots of very smart people enrolled. To get the best mark you
have to be very smart, work very hard, and be a bit lucky. The exam has
to ask just the right questions for your idiosyncratic skills, and you
have to avoid making any dumb mistakes (we all do that sometimes) when
answering them. And that's the thing, whilst intelligence and hard work
are transferable from one class to the next, luck isn't. The people who
got lucky in high school won't be the same as the people who get lucky
at university. That's the very definition of ``luck''. The consequence
of this is that when you select people at the very extreme values of one
measurement (the top 20 students), you're selecting for hard work, skill
and luck. But because the luck doesn't transfer to the second
measurement (only the skill and work), these people will all be expected
to drop a little bit when you measure them a second time (at
university). So their scores fall back a little bit, back towards
everyone else. This is regression to the mean.

Regression to the mean is surprisingly common. For instance, if two very
tall people have kids their children will tend to be taller than average
but not as tall as the parents. The reverse happens with very short
parents. Two very short parents will tend to have short children, but
nevertheless those kids will tend to be taller than the parents. It can
also be extremely subtle. For instance, there have been studies done
that suggested that people learn better from negative feedback than from
positive feedback. However, the way that people tried to show this was
to give people positive reinforcement whenever they did good, and
negative reinforcement when they did bad. And what you see is that after
the positive reinforcement people tended to do worse, but after the
negative reinforcement they tended to do better. But notice that there's
a selection bias here! When people do very well, you're selecting for
``high'' values, and so you should expect, because of regression to the
mean, that performance on the next trial should be worse regardless of
whether reinforcement is given. Similarly, after a bad trial, people
will tend to improve all on their own. The apparent superiority of
negative feedback is an artefact caused by regression to the mean (see
Kahneman \& Tversky (1973), for discussion).

\hypertarget{experimenter-bias}{%
\subsection{Experimenter bias}\label{experimenter-bias}}

\textbf{Experimenter bias} can come in multiple forms. The basic idea is
that the experimenter, despite the best of intentions, can accidentally
end up influencing the results of the experiment by subtly communicating
the ``right answer'' or the ``desired behaviour'' to the participants.
Typically, this occurs because the experimenter has special knowledge
that the participant does not, for example the right answer to the
questions being asked or knowledge of the expected pattern of
performance for the condition that the participant is in. The classic
example of this happening is the case study of ``Clever Hans'', which
dates back to 1907 (Pfungst, 1911). Clever Hans was a horse that
apparently was able to read and count and perform other human like feats
of intelligence. After Clever Hans became famous, psychologists started
examining his behaviour more closely. It turned out that, not
surprisingly, Hans didn't know how to do maths. Rather, Hans was
responding to the human observers around him, because the humans did
know how to count and the horse had learned to change its behaviour when
people changed theirs.

The general solution to the problem of experimenter bias is to engage in
double blind studies, where neither the experimenter nor the participant
knows which condition the participant is in or knows what the desired
behaviour is. This provides a very good solution to the problem, but
it's important to recognise that it's not quite ideal, and hard to pull
off perfectly. For instance, the obvious way that I could try to
construct a double blind study is to have one of my Ph.D.~students (one
who doesn't know anything about the experiment) run the study. That
feels like it should be enough. The only person (me) who knows all the
details (e.g., correct answers to the questions, assignments of
participants to conditions) has no interaction with the participants,
and the person who does all the talking to people (the Ph.D.~student)
doesn't know anything. Except for the reality that the last part is very
unlikely to be true. In order for the Ph.D.~student to run the study
effectively they need to have been briefed by me, the researcher. And,
as it happens, the Ph.D.~student also knows me and knows a bit about my
general beliefs about people and psychology (e.g., I tend to think
humans are much smarter than psychologists give them credit for). As a
result of all this, it's almost impossible for the experimenter to avoid
knowing a little bit about what expectations I have. And even a little
bit of knowledge can have an effect. Suppose the experimenter
accidentally conveys the fact that the participants are expected to do
well in this task. Well, there's a thing called the ``Pygmalion
effect'', where if you expect great things of people they'll tend to
rise to the occasion. But if you expect them to fail then they'll do
that too. In other words, the expectations become a self-fulfilling
prophecy.

\hypertarget{demand-effects-and-reactivity}{%
\subsection{Demand effects and
reactivity}\label{demand-effects-and-reactivity}}

When talking about experimenter bias, the worry is that the
experimenter's knowledge or desires for the experiment are communicated
to the participants, and that these can change people's behaviour
(Rosenthal, 1966). However, even if you manage to stop this from
happening, it's almost impossible to stop people from knowing that
they're part of a psychological study. And the mere fact of knowing that
someone is watching or studying you can have a pretty big effect on
behaviour. This is generally referred to as \textbf{reactivity} or
\textbf{demand effects}. The basic idea is captured by the Hawthorne
effect: people alter their performance because of the attention that the
study focuses on them. The effect takes its name from a study that took
place in the ``Hawthorne Works'' factory outside of Chicago (see Adair
(1984)). This study, from the 1920s, looked at the effects of factory
lighting on worker productivity. But, importantly, change in worker
behaviour occurred because the workers knew they were being studied,
rather than any effect of factory lighting.

To get a bit more specific about some of the ways in which the mere fact
of being in a study can change how people behave, it helps to think like
a social psychologist and look at some of the roles that people might
adopt during an experiment but might not adopt if the corresponding
events were occurring in the real world:

\begin{itemize}
\tightlist
\item
  The \emph{good participant} tries to be too helpful to the researcher.
  He or she seeks to figure out the experimenter's hypotheses and
  confirm them.
\item
  The \emph{negative participant} does the exact opposite of the good
  participant. He or she seeks to break or destroy the study or the
  hypothesis in some way.
\item
  The \emph{faithful participant} is unnaturally obedient. He or she
  seeks to follow instructions perfectly, regardless of what might have
  happened in a more realistic setting.
\item
  The \emph{apprehensive participant} gets nervous about being tested or
  studied, so much so that his or her behaviour becomes highly
  unnatural, or overly socially desirable.
\end{itemize}

\hypertarget{placebo-effects}{%
\subsection{Placebo effects}\label{placebo-effects}}

The \textbf{placebo effect} is a specific type of demand effect that we
worry a lot about. It refers to the situation where the mere fact of
being treated causes an improvement in outcomes. The classic example
comes from clinical trials. If you give people a completely chemically
inert drug and tell them that it's a cure for a disease, they will tend
to get better faster than people who aren't treated at all. In other
words, it is people's belief that they are being treated that causes the
improved outcomes, not the drug.

However, the current consensus in medicine is that true placebo effects
are quite rare and most of what was previously considered placebo effect
is in fact some combination of natural healing (some people just get
better on their own), regression to the mean and other quirks of study
design. Of interest to psychology is that the strongest evidence for at
least some placebo effect is in self-reported outcomes, most notably in
treatment of pain (Hróbjartsson \& Gøtzsche, 2010).

\hypertarget{situation-measurement-and-sub-population-effects}{%
\subsection{Situation, measurement and sub-population
effects}\label{situation-measurement-and-sub-population-effects}}

In some respects, these terms are a catch-all term for ``all other
threats to external validity''. They refer to the fact that the choice
of sub-population from which you draw your participants, the location,
timing and manner in which you run your study (including who collects
the data) and the tools that you use to make your measurements might all
be influencing the results. Specifically, the worry is that these things
might be influencing the results in such a way that the results won't
generalise to a wider array of people, places and measures.

\hypertarget{fraud-deception-and-self-deception}{%
\subsection{Fraud, deception and
self-deception}\label{fraud-deception-and-self-deception}}

\begin{quote}
\emph{It is difficult to get a man to understand something, when his
salary depends on his not understanding it.}\\
-- Upton Sinclair
\end{quote}

There's one final thing I feel I should mention. While reading what the
textbooks often have to say about assessing the validity of a study I
couldn't help but notice that they seem to make the assumption that the
researcher is honest. I find this hilarious. While the vast majority of
scientists are honest, in my experience at least, some are
not.\footnote{Some people might argue that if you're not honest then
  you're not a real scientist. Which does have some truth to it I guess,
  but that's disingenuous (look up the ``No true Scotsman'' fallacy).
  The fact is that there are lots of people who are employed ostensibly
  as scientists, and whose work has all of the trappings of science, but
  who are outright fraudulent. Pretending that they don't exist by
  saying that they're not scientists is just muddled thinking.} Not only
that, as I mentioned earlier, scientists are not immune to belief bias.
It's easy for a researcher to end up deceiving themselves into believing
the wrong thing, and this can lead them to conduct subtly flawed
research and then hide those flaws when they write it up. So you need to
consider not only the (probably unlikely) possibility of outright fraud,
but also the (probably quite common) possibility that the research is
unintentionally ``slanted''. I opened a few standard textbooks and
didn't find much of a discussion of this problem, so here's my own
attempt to list a few ways in which these issues can arise:

\begin{itemize}
\tightlist
\item
  \textbf{Data fabrication}. Sometimes, people just make up the data.
  This is occasionally done with ``good'' intentions. For instance, the
  researcher believes that the fabricated data do reflect the truth, and
  may actually reflect ``slightly cleaned up'' versions of actual data.
  On other occasions, the fraud is deliberate and malicious. Some
  high-profile examples where data fabrication has been alleged or shown
  include Cyril Burt (a psychologist who is thought to have fabricated
  some of his data), Andrew Wakefield (who has been accused of
  fabricating his data connecting the MMR vaccine to autism) and Hwang
  Woo-suk (who falsified a lot of his data on stem cell research).
\item
  \textbf{Hoaxes}. Hoaxes share a lot of similarities with data
  fabrication, but they differ in the intended purpose. A hoax is often
  a joke, and many of them are intended to be (eventually) discovered.
  Often, the point of a hoax is to discredit someone or some field.
  There's quite a few well known scientific hoaxes that have occurred
  over the years (e.g., Piltdown man) and some were deliberate attempts
  to discredit particular fields of research (e.g., the Sokal affair).
\item
  \textbf{Data misrepresentation}. While fraud gets most of the
  headlines, it's much more common in my experience to see data being
  misrepresented. When I say this I'm not referring to newspapers
  getting it wrong (which they do, almost always). I'm referring to the
  fact that often the data don't actually say what the researchers think
  they say. My guess is that, almost always, this isn't the result of
  deliberate dishonesty but instead is due to a lack of sophistication
  in the data analyses. For instance, think back to the example of
  Simpson's paradox that I discussed in the beginning of this book. It's
  very common to see people present ``aggregated'' data of some kind and
  sometimes, when you dig deeper and find the raw data yourself you find
  that the aggregated data tell a different story to the disaggregated
  data. Alternatively, you might find that some aspect of the data is
  being hidden, because it tells an inconvenient story (e.g., the
  researcher might choose not to refer to a particular variable).
  There's a lot of variants on this, many of which are very hard to
  detect.
\item
  \textbf{Study ``misdesign''}. Okay, this one is subtle. Basically, the
  issue here is that a researcher designs a study that has built-in
  flaws and those flaws are never reported in the paper. The data that
  are reported are completely real and are correctly analysed, but they
  are produced by a study that is actually quite wrongly put together.
  The researcher really wants to find a particular effect and so the
  study is set up in such a way as to make it ``easy'' to
  (artefactually) observe that effect. One sneaky way to do this, in
  case you're feeling like dabbling in a bit of fraud yourself, is to
  design an experiment in which it's obvious to the participants what
  they're ``supposed'' to be doing, and then let reactivity work its
  magic for you. If you want you can add all the trappings of double
  blind experimentation but it won't make a difference since the study
  materials themselves are subtly telling people what you want them to
  do. When you write up the results the fraud won't be obvious to the
  reader. What's obvious to the participant when they're in the
  experimental context isn't always obvious to the person reading the
  paper. Of course, the way I've described this makes it sound like it's
  always fraud. Probably there are cases where this is done
  deliberately, but in my experience the bigger concern is with
  unintentional misdesign. The researcher believes and so the study just
  happens to end up with a built-in flaw, and that flaw then magically
  erases itself when the study is written up for publication.
\item
  \textbf{Data mining and post hoc hypothesising}. Another way in which
  the authors of a study can more or less misrepresent the data is by
  engaging in what's referred to as ``data mining'' (see Gelman and
  Loken 2014, for a broader discussion of this as part of the ``garden
  of forking paths'' in statistical analysis). As we'll discuss later,
  if you keep trying to analyse your data in lots of different ways,
  you'll eventually find something that ``looks'' like a real effect but
  isn't. This is referred to as ``data mining''. It used to be quite
  rare because data analysis used to take weeks, but now that everyone
  has very powerful statistical software on their computers it's
  becoming very common. Data mining per se isn't ``wrong'', but the more
  that you do it the bigger the risk you're taking. The thing that is
  wrong, and I suspect is very common, is unacknowledged data mining.
  That is, the researcher runs every possible analysis known to
  humanity, finds the one that works, and then pretends that this was
  the only analysis that they ever conducted. Worse yet, they often
  ``invent'' a hypothesis after looking at the data to cover up the data
  mining. To be clear. It's not wrong to change your beliefs after
  looking at the data, and to reanalyse your data using your new ``post
  hoc'' hypotheses. What is wrong (and I suspect common) is failing to
  acknowledge what you've done. If you acknowledge that you did it then
  other researchers are able to take your behaviour into account. If you
  don't, then they can't. And that makes your behaviour deceptive. Bad!
\item
  \textbf{Publication bias and self-censoring}. Finally, a pervasive
  bias is ``non-reporting'' of negative results. This is almost
  impossible to prevent. Journals don't publish every article that is
  submitted to them. They prefer to publish articles that find
  ``something''. So, if 20 people run an experiment looking at whether
  reading Finnegans Wake causes insanity in humans, and 19 of them find
  that it doesn't, which one do you think is going to get published?
  Obviously, it's the one study that did find that Finnegans Wake causes
  insanity.\footnote{Clearly, the real effect is that only insane people
    would even try to read \emph{Finnegans Wake}.} This is an example of
  a publication bias. Since no-one ever published the 19 studies that
  didn't find an effect, a naive reader would never know that they
  existed. Worse yet, most researchers ``internalise'' this bias and end
  up self-censoring their research. Knowing that negative results aren't
  going to be accepted for publication, they never even try to report
  them. As a friend of mine says ``for every experiment that you get
  published, you also have 10 failures''. And she's right. The catch is,
  while some (maybe most) of those studies are failures for boring
  reasons (e.g., you stuffed something up) others might be genuine
  ``null'' results that you ought to acknowledge when you write up the
  ``good'' experiment. And telling which is which is often hard to do. A
  good place to start is a paper by Ioannidis (2005) with the depressing
  title ``Why most published research findings are false''. I'd also
  suggest taking a look at work by Kühberger et al. (2014) presenting
  statistical evidence that this actually happens in psychology.
\end{itemize}

There's probably a lot more issues like this to think about, but that'll
do to start with. What I really want to point out is the blindingly
obvious truth that real-world science is conducted by actual humans, and
only the most gullible of people automatically assume that everyone else
is honest and impartial. Actual scientists aren't usually that naive,
but for some reason the world likes to pretend that we are, and the
textbooks we usually write seem to reinforce that stereotype.

\hypertarget{summary}{%
\section{Summary}\label{summary}}

This chapter isn't really meant to provide a comprehensive discussion of
psychological research methods. It would require another volume just as
long as this one to do justice to the topic. However, in real life
statistics and study design are so tightly intertwined that it's very
handy to discuss some of the key topics. In this chapter, I've briefly
discussed the following topics:

\begin{itemize}
\tightlist
\item
  \protect\hyperlink{sec-Introduction-to-psychological-measurement}{Introduction
  to psychological measurement}. What does it mean to operationalise a
  theoretical construct? What does it mean to have variables and take
  measurements?
\item
  \protect\hyperlink{sec-Scales-of-measurement}{Scales of measurement}
  and types of variables. Remember that there are two different
  distinctions here. There's the difference between discrete and
  continuous data, and there's the difference between the four different
  scale types (nominal, ordinal, interval and ratio).
\item
  \protect\hyperlink{sec-Assessing-the-reliability-of-a-measurement}{Assessing
  the reliability of a measurement}. If I measure the ``same'' thing
  twice, should I expect to see the same result? Only if my measure is
  reliable. But what does it mean to talk about doing the ``same''
  thing? Well, that's why we have different types of reliability. Make
  sure you remember what they are.
\item
  \protect\hyperlink{the-role-of-variables-predictors-and-outcomes}{The
  ``role'' of variables: predictors and outcomes}. What roles do
  variables play in an analysis? Can you remember the difference between
  predictors and outcomes? Dependent and independent variables? Etc.
\item
  \protect\hyperlink{experimental-and-non-experimental-research}{Experimental
  and non-experimental research} designs. What makes an experiment an
  experiment? Is it a nice white lab coat, or does it have something to
  do with researcher control over variables?
\item
  \protect\hyperlink{assessing-the-validity-of-a-study}{Assessing the
  validity of a study}. Does your study measure what you want it to? How
  might things go wrong? And is it my imagination, or was that a very
  long list of possible ways in which things can go wrong?
\end{itemize}

All this should make clear to you that study design is a critical part
of research methodology. I built this chapter from the classic little
book by Campbell \& Stanley (1963), but there are of course a large
number of textbooks out there on research design. Spend a few minutes
with your favourite search engine and you'll find dozens.

\part{An introduction to jamovi}

\hypertarget{getting-started-with-jamovi}{%
\chapter{Getting started with
jamovi}\label{getting-started-with-jamovi}}

\placetextbox{0.26}{0.06}{\scriptsize{©2025 D. Foxcroft and D. Navarro,}}
\placetextbox{0.205}{0.05}{\scriptsize{CC BY-SA 4.0}}
\placetextbox{0.70}{0.06}{\scriptsize{\url{https://doi.org/10.11647/OBP.0333/03}}}

\begin{quote}
\emph{Robots are nice to work with.}\\
-- Roger Zelazny\footnote{Source: \emph{Dismal Light} (1968).}
\end{quote}

In this chapter I'll discuss how to get started in jamovi. I'll briefly
talk about how to download and install jamovi, but most of the chapter
will be focused on getting you started with finding your way around the
jamovi graphical user interface (GUI). Our goal in this chapter is not
to learn any statistical concepts: we're just trying to learn the basics
of how jamovi works and get comfortable interacting with the system. To
do this we'll spend a bit of time looking at data sets and variables. In
doing so, you'll get something of a feel for what it's like to work in
jamovi.

However, before going into any of the specifics, it's worth talking a
little about why you might want to use jamovi at all. Given that you're
reading this you've probably got your own reasons. However, if those
reasons are ``because that's what my stats class uses'', it might be
worth explaining a little why your lecturer has chosen to use jamovi for
the class. Of course, I don't really know why \emph{other} people choose
jamovi so I'm really talking about why I use it.

\begin{itemize}
\tightlist
\item
  It's sort of obvious but worth saying anyway: doing your statistics on
  a computer is faster, easier and more powerful than doing statistics
  by hand. Computers excel at mindless repetitive tasks, and a lot of
  statistical calculations are both mindless and repetitive. For most
  people the only reason to ever do statistical calculations with pencil
  and paper is for learning purposes. In my class I do occasionally
  suggest doing some calculations that way, but the only real value to
  it is pedagogical. It does help you to get a ``feel'' for statistics
  to do some calculations yourself, so it's worth doing it once. But
  only once!
\item
  Doing statistics in a conventional spreadsheet (e.g., Microsoft Excel)
  is generally a bad idea in the long run. Although many people likely
  feel more familiar with them, spreadsheets are very limited in terms
  of what analyses they allow you do. If you get into the habit of
  trying to do your real-life data analysis using spreadsheets then
  you've dug yourself into a very deep hole.
\item
  Avoiding proprietary software is a very good idea. There are a lot of
  commercial packages out there that you can buy, some of which I like
  and some of which I don't. They're usually very glossy in their
  appearance, and generally very powerful (much more powerful than
  spreadsheets). However, they're also very expensive: usually, the
  company sells ``student versions'' (crippled versions of the real
  thing) very cheaply; they sell full powered ``educational versions''
  at a price that makes me wince; and they sell commercial licences with
  a staggeringly high price tag. The business model here is to suck you
  in during your student days and then leave you dependent on their
  tools when you go out into the real world. It's hard to blame them for
  trying, but personally I'm not in favour of shelling out thousands of
  dollars if I can avoid it. And you can avoid it. If you make use of
  packages like jamovi that are open source and free you never get
  trapped having to pay exorbitant licensing fees.
\item
  Something that you might not appreciate now, but will love later on if
  you do anything involving data analysis, is the fact that jamovi is
  basically a sophisticated front end for the free R statistical
  programming language. When you download and install R you get all the
  basic ``packages'' and those are very powerful on their own. However,
  because R is so open and so widely used, it's become something of a
  standard tool in statistics and so lots of people write their own
  packages that extend the system. And these are freely available too.
  One of the consequences of this, I've noticed, is that if you look at
  recent advanced data analysis textbooks then a lot of them use R.
\end{itemize}

Those are the main reasons I use jamovi. It's not without its flaws,
though. It's relatively new\footnote{At the time of first writing this
  in August 2018. Later versions of this book will use later versions of
  jamovi.} so there is not a huge set of textbooks and other resources
to support it, and it has a few annoying quirks that we're all pretty
much stuck with, but on the whole I think the strengths outweigh the
weakness; more so than any other option I've encountered so far.

\hypertarget{installing-jamovi}{%
\section{Installing jamovi}\label{installing-jamovi}}

Okay, enough with the sales pitch. Let's get started. Just as with any
piece of software, jamovi needs to be installed on a ``computer'', which
is a magical box that does cool things and delivers free ponies. Or
something along those lines; I may be confusing computers with the iPad
marketing campaigns. Anyway, jamovi is freely distributed online and you
can download it from the jamovi homepage, which is
\url{https://www.jamovi.org/}.

At the top of the page under the heading ``Download'', you'll see
separate links for Windows users, Mac users and Linux users. If you
follow the relevant link you'll see that the online instructions are
pretty self-explanatory. At the time of writing, the current version of
jamovi is 2.3, but they usually issue updates every few months, so
you'll probably have a newer version.\footnote{Although jamovi is
  updated frequently it doesn't usually make much of a difference for
  the sort of work we'll do in this book. In fact, during the writing of
  the book I upgraded several times and it didn't make much difference
  at all to what is in this book.}

\hypertarget{starting-up-jamovi}{%
\subsection{Starting up jamovi}\label{starting-up-jamovi}}

One way or another, regardless of what operating system you're using,
it's time to open jamovi and get started. When first starting jamovi you
will be presented with a user interface which looks something like
Figure~\ref{fig-fig3-1}.

\begin{figure}

\includegraphics[width=1\textwidth,height=\textheight]{images/fig3-1.png} \hfill{}

\caption{\label{fig-fig3-1}jamovi starts up!}

\end{figure}

To the left is the spreadsheet view, and to the right is where the
results of statistical tests appear. Down the middle is a bar separating
these two regions and this can be dragged to the left or the right to
change their sizes.

It is possible to simply begin typing values into the jamovi spreadsheet
as you would in any other spreadsheet software. Alternatively, existing
data sets in the csv (.csv) file format can be opened in jamovi.
Additionally, you can easily import SPSS, SAS, STATA and JASP files
directly into jamovi. To open a file select the `File'\footnote{From now
  on, we'll use single quote marks to signify a label, command, option,
  or outputs in the jamovi interface.} tab (three horizontal lines
signify this tab) at the top left hand corner, select `Open' and then
choose from the files listed on `Browse' depending on whether you want
to open an example or a file stored on your computer.

\hypertarget{analyses}{%
\section{Analyses}\label{analyses}}

Analyses can be selected from the analysis ribbon or menu along the top.
Selecting an analysis will present an `Options panel' for that
particular analysis, allowing you to assign different variables to
different parts of the analysis, and select different options. At the
same time, the results for the analysis will appear in the right
`Results panel' and will update in real time as you make changes to the
options.

When you have the analysis set up correctly you can dismiss the analysis
options by clicking the arrow to the top right of the optional panel. If
you wish to return to these options, you can click on the results that
were produced. In this way, you can return to any analysis that you (or
say, a colleague) created earlier.

If you decide you no longer need a particular analysis, you can remove
it with the results context menu. Right-clicking on the analysis results
will bring up a menu and by selecting `Analysis' and then `Remove' the
analysis can be removed. But more on this later. First, let's take a
more detailed look at the spreadsheet view.

\hypertarget{the-spreadsheet}{%
\section{The spreadsheet}\label{the-spreadsheet}}

In jamovi data is represented in a spreadsheet with each column
representing a `variable' and each row representing a `case' or
`participant'.

\hypertarget{variables}{%
\subsection{Variables}\label{variables}}

The most commonly used variables in jamovi are `Data variables', these
variables simply contain data either loaded from a data file, or `typed
in' by the user. Data variables can be one of several measurement levels
(Figure~\ref{fig-fig3-2}).

\begin{figure}

\includegraphics[width=1\textwidth,height=\textheight]{images/fig3-2.png} \hfill{}

\caption{\label{fig-fig3-2}measurement levels}

\end{figure}

These levels are designated by the symbol in the header of the
variable's column. The ID variable type is unique to jamovi. It's
intended for variables that contain identifiers that you would almost
never want to analyse. For example, a person's name, or a participant
ID. Specifying an ID variable type can improve performance when
interacting with very large data sets.

\emph{Nominal} variables are for categorical variables which are text
labels, for example a column called `gender' with the values `male' and
`female' would be nominal. So would a person's name. Nominal variable
values can also have a numeric value. These variables are used most
often when importing data which codes values with numbers rather than
text. For example, a column in a data set may contain the values 1 for
males, and 2 for females. It is possible to add nice `human-readable'
labels to these values with the variable editor (more on this later).

\emph{Ordinal} variables are like nominal variables, except the values
have a specific order. An example is a Likert scale with 3 being
`strongly agree' and -3 being `strongly disagree'.

\emph{Continuous} variables are variables which exist on a continuous
scale. Examples might be height or weight. This is also referred to as
`Interval' or `Ratio scale'.

In addition, you can also specify different data types: variables have a
data type of either `Text', `Integer' or `Decimal'.

When starting with a blank spreadsheet and typing values in the variable
type will change automatically depending on the data you enter. This is
a good way to get a feel for which variable types go with which sorts of
data. Similarly, when opening a data file jamovi will try and guess the
variable type from the data in each column. In both cases this automatic
approach may not be correct, and it may be necessary to manually specify
the variable type with the variable editor.

The variable editor can be opened by selecting `Setup' from the data tab
or by double-clicking on the variable column header. The variable editor
allows you to change the name of the variable and, for data variables,
the variable type, the order of the levels, and the label displayed for
each level. Changes can be applied by clicking the `tick' to the top
right. The variable editor can be dismissed by clicking the `Hide'
arrow.

New variables can be inserted or appended to the data set using the
`Add' button from the data ribbon. The `Add' button also allows the
addition of computed variables.

\hypertarget{computed-variables}{%
\subsection{Computed variables}\label{computed-variables}}

Computed Variables are those which take their value by performing a
computation on other variables. Computed Variables can be used for a
range of purposes, including log transforms, \emph{z}-scores,
sum-scores, negative scoring and means.

Computed variables can be added to the data set with the `Add' button
available on the data tab. This will produce a formula box where you can
specify the formula. The usual arithmetic operators are available. Some
examples of formulas are:

A + B LOG10(len) MEAN(A, B) (len - VMEAN(len)) / VSTDEV(len)

In order, these are the sum of A and B, a log (base 10) transform of
len, the mean of A and B, and the \emph{z}-score of the variable
len.\footnote{In later versions of jamovi there is a pre-defined
  function `Z' to compute \emph{z}-scores, which is much easier!}
Figure~\ref{fig-fig3-3} shows the jamovi screen for the new variable
computed as the \emph{z}-score of len (from the `Tooth Growth' example
data set).

\begin{figure}

\includegraphics[width=1\textwidth,height=\textheight]{images/fig3-3.png} \hfill{}

\caption{\label{fig-fig3-3}A newly computed variable, the \emph{z}-score
of `dose'}

\end{figure}

\hypertarget{v-functions}{%
\subsubsection{V-functions}\label{v-functions}}

Several functions are already available in jamovi and available from the
drop down box labelled fx. A number of functions appear in pairs, one
prefixed with a V and the other not. V functions perform their
calculation on a variable as a whole, where as non-V functions perform
their calculation row by row. For example, MEAN(A, B) will produce the
mean of A and B for each row. Where as VMEAN(A) gives the mean of all
the values in A.

\hypertarget{copy-and-paste}{%
\subsection{Copy and paste}\label{copy-and-paste}}

jamovi produces nice American Psychological Association (APA) formatted
tables and attractive plots. It is often useful to be able to copy and
paste these, perhaps into a Word document, or into an email to a
colleague. To copy results right click on the object of interest and
from the menu select exactly what you want to copy. The menu allows you
to choose to copy only the image or the entire analysis. Selecting
`Copy' copies the content to the clipboard and this can be pasted into
other programs in the usual way. You can practice this later on when we
do some analyses.

\hypertarget{syntax-mode}{%
\subsection{Syntax mode}\label{syntax-mode}}

jamovi also provides an `R Syntax mode'.\footnote{\(R\) is a powerful
  statistical programming language. In fact, jamovi is just a
  user-friendly interface that sits on top of the \(R\) engine.} In this
mode jamovi produces equivalent \(R\) code for each analysis. To change
to syntax mode, select the `Application' menu to the top right of jamovi
(a button with three vertical dots) and click the `Syntax mode' checkbox
there. You can turn off syntax mode by clicking this a second time.

In syntax mode analyses continue to operate as before but now they
produce \(R\) syntax, and ``ascii output'' like an \(R\) session. Like
all results objects in jamovi, you can right click on these items
(including the \(R\) syntax) and copy and paste them, for example into
an \(R\) session. At present, the provided \(R\) syntax does not include
the data import step and so this must be performed manually in \(R\).
There are many resources explaining how to import data into \(R\) and if
you are interested we recommend you take a look at these; just search on
the interweb.

\hypertarget{loading-data-in-jamovi}{%
\section{Loading data in jamovi}\label{loading-data-in-jamovi}}

There are several different types of files that are likely to be
relevant to us when doing data analysis. There are two in particular
that are especially important from the perspective of this book:

\begin{itemize}
\item
  \emph{jamovi files} are those with a .omv file extension. This is the
  standard kind of file that jamovi uses to store data, and variables
  and analyses.
\item
  \emph{Comma separated value (csv) files} are those with a .csv file
  extension. These are just regular old text files and they can be
  opened with many different software programs. It's quite typical for
  people to store data in csv files, precisely because they're so
  simple.
\end{itemize}

There are also several other kinds of data file that you might want to
import into jamovi. For instance, you might want to open Microsoft Excel
spreadsheets (.xls files), or data files that have been saved in the
native file formats for other statistics software, such as SPSS or SAS.
Whichever file formats you are using, it's a good idea to create a
folder or folders especially for your jamovi data sets and analyses and
to make sure you keep these backed up regularly.

\hypertarget{importing-data-from-csv-files}{%
\subsection{Importing data from csv
files}\label{importing-data-from-csv-files}}

One quite commonly used data format is the humble ``comma separated
value'' file, also called a csv file, and usually bearing the file
extension .csv. csv files are just plain old-fashioned text files and
what they store is basically just a table of data. This is illustrated
in Figure~\ref{fig-fig3-4}, which shows a file called
\emph{booksales.csv} that I've created. As you can see, each row
represents the book sales data for one month. The first row doesn't
contain actual data though, it has the names of the variables.

\begin{figure}

\includegraphics[width=0.8\textwidth,height=\textheight]{images/fig3-4.png} \hfill{}

\caption{\label{fig-fig3-4}The \emph{booksales.csv} data file. On the
left I have opened the file using a spreadsheet program, which shows
that the file is basically a table. On the right the same file is open
in a standard text editor (the TextEdit program on a Mac), which shows
how the file is formatted. The entries in the table are separated by
commas}

\end{figure}

It's easy to open csv files in jamovi. From the top left menu (the
button with three parallel lines) choose `Open' and browse to where you
have stored the csv file on your computer. If you're on a Mac, it'll
look like the usual Finder window that you use to choose a file; on
Windows it looks like an Explorer window. An example of what it looks
like on a Mac is shown in Figure~\ref{fig-fig3-5}. I'm assuming that
you're familiar with your own computer, so you should have no problem
finding the csv file that you want to import! Find the one you want,
then click on the `Open' button.

\begin{figure}

\includegraphics[width=0.8\textwidth,height=\textheight]{images/fig3-5.png} \hfill{}

\caption{\label{fig-fig3-5}A dialog box on a Mac asking you to select
the csv file jamovi should try to import. Mac users will recognise this
immediately, as it is the usual way in which a Mac asks you to find a
file. Windows users will not see this, instead they will see the usual
explorer window that Windows always gives you when it wants you to
select a file}

\end{figure}

There are a few things that you can check to make sure that the data
gets imported correctly:

\begin{itemize}
\tightlist
\item
  Heading. Does the first row of the file contain the names for each
  variable -- a `header' row? The \emph{booksales.csv} file has a
  header, so that's a yes.
\item
  Decimal. What character is used to specify the decimal point? In
  English-speaking countries this is almost always a period (i.e., .).
  That's not universally true though, many European countries use a
  comma.
\item
  Quote. What character is used to denote a block of text? That's
  usually going to be a double quote mark (``). It is for the
  \emph{booksales.csv} file.
\end{itemize}

\hypertarget{importing-unusual-data-files}{%
\section{Importing unusual data
files}\label{importing-unusual-data-files}}

Throughout this book I've assumed that your data are stored as a jamovi
.omv file or as a ``properly'' formatted csv file. However, in real life
that's not a terribly plausible assumption to make so I'd better talk
about some of the other possibilities that you might run into.

\hypertarget{loading-data-from-text-files}{%
\subsection{Loading data from text
files}\label{loading-data-from-text-files}}

The first thing I should point out is that if your data are saved as a
text file but aren't quite in the proper csv format then there's still a
pretty good chance that jamovi will be able to open it. You just need to
try it and see if it works. Sometimes though you will need to change
some of the formatting. The ones that I've often found myself needing to
change are:

\begin{itemize}
\tightlist
\item
  header. A lot of the time when you're storing data as a csv file the
  first row actually contains the column names and not data. If that's
  not true then it's a good idea to open up the csv file in a
  spreadsheet programme such as Open Office and add the header row
  manually.
\item
  sep. As the name ``comma separated value'' indicates, the values in a
  row of a csv file are usually separated by commas. This isn't
  universal, however. In Europe the decimal point is typically written
  as , instead of . and as a consequence it would be somewhat awkward to
  use , as the separator. Therefore it is not unusual to use ; instead
  of , as the separator. At other times, I've seen a TAB character used.
\item
  quote. It's conventional in csv files to include a quoting character
  for textual data. As you can see by looking at the
  \emph{booksales.csv} file, this is usually a double quote character,
  ``. But sometimes there is no quoting character at all, or you might
  see a single quote mark ' used instead.
\item
  skip. It's actually very common to receive csv files in which the
  first few rows have nothing to do with the actual data. Instead, they
  provide a human readable summary of where the data came from, or maybe
  they include some technical info that doesn't relate to the data.
\item
  missing values. Often you'll get given data with missing values. For
  one reason or another, some entries in the table are missing. The data
  file needs to include a ``special'' value to indicate that the entry
  is missing. By default jamovi assumes that this value is
  99,\footnote{You can change the default value for missing values in
    jamovi from the top right menu (three vertical dots), but this only
    works at the time of importing data files into jamovi. The default
    missing value in the data set should not be a valid number
    associated with any of the variables, e.g., you could use -9999 as
    this is unlikely to be a valid value.} for both numeric and text
  data, so you should make sure that, where necessary, all missing
  values in the csv file are replaced with 99 (or -9999; whichever you
  choose) before opening / importing the file into jamovi. Once you have
  opened / imported the file into jamovi all the missing values are
  converted to blank or greyed out cells in the jamovi spreadsheet view.
  You can also change the missing value for each variable as an option
  in the Data - Setup view.
\end{itemize}

\hypertarget{loading-data-from-spss-and-other-statistics-packages}{%
\subsection{Loading data from SPSS (and other statistics
packages)}\label{loading-data-from-spss-and-other-statistics-packages}}

The commands listed above are the main ones we'll need for data files in
this book. But in real life we have many more possibilities. For
example, you might want to read data files in from other statistics
programs. Since SPSS is probably the most widely used statistics package
in psychology, it's worth mentioning that jamovi can also import SPSS
data files (file extension .sav). Just follow the instructions above for
how to open a csv file, but this time navigate to the .sav file you want
to import. For SPSS files, jamovi will regard all values as missing if
they are regarded as ``system missing'' files in SPSS. The `Default
missings' value does not seem to work as expected when importing SPSS
files, so be aware of this -- you might need another step: import the
SPSS file into jamovi, then export as a csv file before re-opening in
jamovi.\footnote{I know this is a bit of a fudge, but it does work and
  hopefully this will be fixed in a later version of jamovi.}

And that's pretty much it, at least as far as SPSS goes. As far as other
statistical software goes, jamovi can also directly open / import SAS
and STATA files.

\hypertarget{loading-excel-files}{%
\subsection{Loading Excel files}\label{loading-excel-files}}

A different problem is posed by Excel files. Despite years of yelling at
people for sending data to me encoded in a proprietary data format, I
get sent a lot of Excel files. The way to handle Excel files is to open
them up first in Excel or another spreadsheet programme that can handle
Excel files, and then export the data as a csv file before opening /
importing the csv file into jamovi.

\hypertarget{sec-Changing-data-from-one-level-to-another}{%
\section{Changing data from one level to
another}\label{sec-Changing-data-from-one-level-to-another}}

Sometimes you want to change the variable level. This can happen for all
sorts of reasons. Sometimes when you import data from files, it can come
to you in the wrong format. Numbers sometimes get imported as nominal,
text values. Dates may get imported as text. Participant ID values can
sometimes be read as continuous: nominal values can sometimes be read as
ordinal or even continuous. There's a good chance that sometimes you'll
want to convert a variable from one measurement level into another one.
Or, to use the correct term, you want to \textbf{coerce} the variable
from one class into another.

Earlier we saw how to specify different variable levels, and if you want
to change a variable's measurement level then you can do this in the
jamovi data view for that variable. Just click the check box for the
measurement level you want -- continuous, ordinal, or nominal.

\hypertarget{installing-add-on-modules-into-jamovi}{%
\section{Installing add-on modules into
jamovi}\label{installing-add-on-modules-into-jamovi}}

A really great feature of jamovi is the ability to install add-on
modules from the jamovi library. These add-on modules have been
developed by the jamovi community, i.e., jamovi users and developers who
have created special software add-ons that do other, usually more
advanced, analyses that go beyond the capabilities of the base jamovi
program.

To install add-on modules, just click on the large \(+\) in the top
right of the jamovi window, select ``jamovi-library'' and then browse
through the various add-on modules that are available. Choose the one(s)
you want, and then install them, as in Figure~\ref{fig-fig3-6}. It's
that easy. The newly installed modules can then be accessed from the
``Analyses'' button bar. Try it\ldots useful add-on modules to install
include ``scatr'' (added under ``Descriptives''), \(R_j\) and \textbf{of
course} the data files for this book: ``lsj-data''.

\begin{figure}

\includegraphics[width=1\textwidth,height=\textheight]{images/fig3-6.png} \hfill{}

\caption{\label{fig-fig3-6}Installing add-on modules in jamovi}

\end{figure}

\hypertarget{quitting-jamovi}{%
\section{Quitting jamovi}\label{quitting-jamovi}}

There's one last thing I should cover in this chapter: how to quit
jamovi. It's not hard, just close the program the same way you would any
other program. However, what you might want to do before you quit is
save your work! There are two parts to this: saving any changes to the
data set, and saving the analyses that you ran.

It is good practice to save any changes to the data set as a \emph{new}
data set. That way you can always go back to the original data. To save
any changes in jamovi, select `Export'\ldots{}`Data' from the main
jamovi menu (button with three horizontal bars in the top left) and
create a new file name for the changed data set.

Alternatively, you can save \emph{both} the changed data and any
analyses you have undertaken by saving as a jamovi file. To do this,
from the main jamovi menu select `Save as' and type in a file name for
this `jamovi file (.omv)'. Remember to save the file in a location where
you can find it again later. I usually create a new folder for specific
data sets and analyses.

\hypertarget{summary-1}{%
\section{Summary}\label{summary-1}}

Every book that tries to teach a new statistical software program to
novices has to cover roughly the same topics, and in roughly the same
order. Ours is no exception, and so in the grand tradition of doing it
just the same way everyone else did it, this chapter covered the
following topics:

\begin{itemize}
\tightlist
\item
  \protect\hyperlink{installing-jamovi}{Installing jamovi}. We
  downloaded and installed jamovi, and started it up.
\item
  \protect\hyperlink{analyses}{Analyses}. We very briefly oriented to
  the part of jamovi where analyses are done and results appear, but
  then deferred this until later in the book.
\item
  \protect\hyperlink{the-spreadsheet}{The spreadsheet}. We spent more
  time looking at the spreadsheet part of jamovi, and considered
  different variable types, and how to compute new variables.
\item
  \protect\hyperlink{loading-data-in-jamovi}{Loading data in jamovi}. We
  also saw how to load data files in jamovi.
\item
  \protect\hyperlink{importing-unusual-data-files}{Importing unusual
  data files}. Then we figured out how to open other data files, from
  different file types.
\item
  \protect\hyperlink{sec-Changing-data-from-one-level-to-another}{Changing
  data from one level to another}. And saw that sometimes we need to
  coerce data from one type to another.
\item
  \protect\hyperlink{installing-add-on-modules-into-jamovi}{Installing
  add-on modules into jamovi}. Installing add-on modules from the jamovi
  community really extends jamovi capabilities.
\item
  \protect\hyperlink{quitting-jamovi}{Quitting jamovi}. Finally, we
  looked at good practice in terms of saving your data set and analyses
  when you have finished and are about to quit jamovi.
\end{itemize}

We still haven't arrived at anything that resembles data analysis. Maybe
the next chapter will get us a bit closer!

\hypertarget{refs}{}
\begin{CSLReferences}{1}{0}
\leavevmode\vadjust pre{\hypertarget{ref-Adair1984}{}}%
Adair, G. (1984). The hawthorne effect: A reconsideration of the
methodological artifact. \emph{Journal of Applied Psychology},
\emph{69}, 334--345. \url{https://doi.org/10.1037/0021-9010.69.2.334}

\leavevmode\vadjust pre{\hypertarget{ref-Bickel1975}{}}%
Bickel, P. J., Hammel, E. A., \& O'Connell, J. W. (1975). Sex bias in
graduate admissions: Data from {B}erkeley. \emph{Science}, \emph{187},
398--404. \url{https://doi.org/10.1126/science.187.4175.398}

\leavevmode\vadjust pre{\hypertarget{ref-Campbell1963}{}}%
Campbell, D. T., \& Stanley, J. C. (1963). \emph{Experimental and
quasi-experimental designs for research}. Houghton Mifflin.

\leavevmode\vadjust pre{\hypertarget{ref-Evans1983}{}}%
Evans, J. St. B. T., Barston, J. L., \& Pollard, P. (1983). On the
conflict between logic and belief in syllogistic reasoning. \emph{Memory
and Cognition}, \emph{11}, 295--306.
\url{https://doi.org/10.3758/BF03196976}

\leavevmode\vadjust pre{\hypertarget{ref-hrobjartsson2010}{}}%
Hróbjartsson, A., \& Gøtzsche, P. (2010). Placebo interventions for all
clinical conditions. \emph{Cochrane Database of Systematic Reviews},
\emph{1}. \url{https://doi.org/10.1002/14651858.cd003974.pub3}

\leavevmode\vadjust pre{\hypertarget{ref-Ioannidis2005}{}}%
Ioannidis, J. P. A. (2005). Why most published research findings are
false. \emph{PLoS Med}, \emph{2}(8), 697--701.
\url{https://doi.org/10.1371/journal.pmed.1004085}

\leavevmode\vadjust pre{\hypertarget{ref-Kahneman1973}{}}%
Kahneman, D., \& Tversky, A. (1973). On the psychology of prediction.
\emph{Psychological Review}, \emph{80}, 237--251.
\url{https://doi.org/10.1037/h0034747}

\leavevmode\vadjust pre{\hypertarget{ref-Kuhberger2014}{}}%
Kühberger, A., Fritz, A., \& Scherndl, T. (2014). Publication bias in
psychology: A diagnosis based on the correlation between effect size and
sample size. \emph{Public Library of Science One}, \emph{9}, 1--8.
\url{https://doi.org/10.1371/journal.pone.0105825}

\leavevmode\vadjust pre{\hypertarget{ref-Pfungst1911}{}}%
Pfungst, O. (1911). \emph{Clever hans (the horse of mr. Von osten): A
contribution to experimental animal and human psychology} (C. L. Rahn,
Trans.). Henry Holt.

\leavevmode\vadjust pre{\hypertarget{ref-Rosenthal1966}{}}%
Rosenthal, R. (1966). \emph{Experimenter effects in behavioral
research}. Appleton.

\leavevmode\vadjust pre{\hypertarget{ref-Stevens1946}{}}%
Stevens, S. S. (1946). On the theory of scales of measurement.
\emph{Science}, \emph{103}, 677--680.
\url{https://doi.org/10.1126/science.103.2684.677}

\end{CSLReferences}


\backmatter

\printendnotes
\newpage

\chapter{About the team}

Alessandra Tosi was the managing editor for this book.

Tricia de Souza and Adèle Kreager proof-read this manuscript.

The cover was designed by Jeevanjot Kaur Nagpal, and produced in InDesign using the Fontin and Calibri fonts.

David Foxcroft and Cameron Craig produced the printed PDF editions. 

David Foxcroft produced the HTML edition.

Raegan Allen was in charge of marketing.

This book was peer-reviewed by two referees. Experts in their field, these readers give their time freely to help ensure the academic rigour of our books. We are grateful for their generous and invaluable contributions.

\newpage


%\printindex
%back cover
%\includepdf[fitpaper=true,pages=-]{images/backcover8x10}
%\pagenumbering{gobble} 

\end{document}
